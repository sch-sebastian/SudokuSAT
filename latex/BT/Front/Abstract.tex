% !TEX root = ../Thesis.tex
\chapter{Abstract}
Sudoku has become one of the world's most popular logic puzzles, arousing interest in the general public and the science community. Although the rules of Sudoku may seem simple, they allow for nearly countless puzzle instances, some of which are very hard to solve. SAT-solvers have proven to be a suitable option to solve Sudokus automatically. However, they demand the puzzles to be encoded as logical formulae in Conjunctive Normal Form. In earlier work, such encodings have been successfully demonstrated for original Sudoku Puzzles. In this thesis, we present encodings for rather unconventional Sudoku Variants, developed by the puzzle community to create even more challenging solving experiences. Furthermore, we demonstrate how Pseudo-Boolean Constraints can be utilized to encode Sudoku Variants that follow rules involving sums. To implement an encoding of Pseudo-Boolean Constraints, we use Binary Decision Diagrams and Adder Networks and study how they compare to each other.
