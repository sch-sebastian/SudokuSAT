% !TEX root = ../Thesis.tex
\chapter{Conclusion}
This thesis demonstrated how different variants and rules of Sudoku Puzzles could be encoded as sets of logical clauses. We have seen that in most cases, SAT-solvers can find assignments that satisfy these sets of clauses in a relatively short time. We also found that solving Sudokus is by no means a trivial problem as the number of clauses and the time needed to solve ``Nurikabe Sudoku'' have shown. We have elaborated in detail on how Pseudo Boolean Constraints can be used to encode constraints regarding sums, like in Killer Sudokus or for the Sandwich-Sum rules. Specifically for Killer Sudoku instances, we found that both shown encoding methods Adder Networks and Binary Decision Diagrams (both proposed by \cite{Een2006TranslatingPC}) have comparable performance, but that Adder Networks produce fewer clauses and are encoding the PBCs faster. Additionally, we have shown that encoding Killer Sudokus without PBCs can significantly reduce the time needed to solve them.\\
\\
In future work, the proposed encoding methods for Sudoku rules could help to craft new puzzle instances of exotic Variants like ``9 Marks The Spot'' or ``The Miracle Thermo''. Further, we came across multiple engaging questions for which we only estimated an answer or gave an upper bound, like what is the highest possible number of islands in ``Nurikabe Sudoku" or how long can a hidden path in ``9 Marks The Spot'' be at maximum. Also teasing: CTCGH \cite{CrackingTheCryptic2021} contains many more Sudoku variants which could be analysed and encoded for SAT-Solvers.\\
\\
During our work, we only used a limited test set of puzzle instances, which all had to be rewritten by hand into a format our program could understand. In general, there seem to be no common test sets for non-original Sudoku Variants like Killer Sudoku, which may also has to do with the lack of a common file format that could be used to share more exotic Puzzle variants. To facilitate future work, it would be desirable to agree on a standard file format specifically tailored for Sudokus. This would allow the puzzle and research community to build an openly available and directly usable database of Sudoku Puzzles, which would make the testing and analysis of new encoding and solving algorithms more reliable.