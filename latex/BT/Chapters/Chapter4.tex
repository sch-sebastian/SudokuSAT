% !TEX root = ../Thesis.tex
\chapter{Encoding}\label{Encoding}
As introduced in \ref{DIMACS}, the SAT-solvers expect a DIMACS file as input that describes a set of clauses. This chapter details how the different Sudoku Variants and their rules can be encoded into these clause sets. We elaborate on the different variants separately, but as long as there are no direct contradictions, the different variants and rules could be freely combined (which can be done by creating the union of the corresponding clause sets). In the following formulae, we use the notation $s_{x_{n},...,x_{0}}$ to describe boolean variables, the corresponding integer numbers in the DIMACS files have the values $x_{n}*10^{n}+...+x_{0}*10^{0}$. For example, the literal $\neg s_{1,2,3}$ would be transformed to $-123$. There are two ways how we choose the name (number) for a variable during the encoding process:

\begin{enumerate}
    \item We use increasing values starting from 1. These dynamically chosen values are, by example, used for variables necessary to encode PBCs. We use $s_{v}$, for $v \in \mathbb{N}$, to describe them in the formulae as we do not know the corresponding integers in advance.
    \item  We use fixed intervals of numbers to encode certain constraints. Here every digit of a number has a semantic meaning. The dynamically generated variables skip these fixed intervals. For example, a fixed interval is used for the boolean variables that describe the cell values of the Sudoku grid. The variable $s_{x,y,z}$ is true if and only if cell ($x$,$y$) has the value $z$ assigned (as proposed in \cite{Lynce2006SudokuAsASATProblem}). The other used fixed intervals are indicated in the corresponding sections of this chapter. 
\end{enumerate}

It is important to note that the numbers used as variable names are not ``continuous". There are ``gaps". For example, not all integers from 1 to 1000 are used. This is because the dynamic variables skip the entire interval from 111 to 999, and the fixed variables used to encode cell values will not use numbers like 400 because we start to count rows at 1, and the grid cells can only hold values from 1 to 9. The needed fixed intervals grow larger when encoding more complicated variants, and with them often also the ``gaps". These gaps have no effect logically, but they can have a non-negligible effect on the time needed by the solvers, which is affected by the highest integer value used to describe a variable. However, as the ``gaps" are the same when encoding Sudokus with the same rules, they should not make a difference when comparing solver-times.

\section{Encoding of PBCs}\label{encoding:PBCs}
In the following two subsections (\ref{PBCEncodingBDD} and \ref{PBCEncodingAdderNetworks}), we will elaborate on how PBCs can be encoded into clauses. As teased earlier, we will follow the ideas of \cite{Een2006TranslatingPC} and use Binary Decision Diagrams and Adder Networks to perform this translation. The two methods approach their task rather differently, so we will compare their results and performance in chapter \ref{Experiments}.

\subsection{Encoding of PBCs using Binary Decision Diagrams}\label{PBCEncodingBDD}
A binary decision diagram (BDD) is a directed graph that can be used to represent logical formulae and PBCs. 
Variables are considered in a fixed order. As discussed in \cite{Een2006TranslatingPC}, ordering them by decreasing weight values is generally reasonable. Every graph node corresponds to a sum and a subset of variables of the formula. The BDD has a root node corresponding to the empty variable set and sum 0. If a node is $n$ edges away from the root, its variable set contains the $n$ first variables, and such a node is said to be at depth $n$. An edge from a node $u$ at depth $k$ to a node $v$ at depth $k+1$ corresponds to assigning a truth value to the $(k+1)$-th variable. Every node has at most two successors, one reachable via the edge that corresponds to assigning \true{} and one via the edge that corresponds to assigning \false{} to the next variable. Nodes store references to their successors as \emph{$\textit{true}_{child}$} and \emph{$\textit{false}_{child}$}, respectively. Paths from the root to a node correspond to variable assignments to the variables in a node´s variable set and define the node's sum, which is equal to the sum of PBC weights that are multiplied by variables that are set to \true{} by the assignment.\\

Terminal nodes however have no children. Terminal nodes are either nodes at depth $l$ (with $l$ equal to the total number of different variables in the formula) or nodes with a too high or too low sum regarding the variables already assigned in the paths to them.\\

An integer number is assigned to each node which can later be used to describe a boolean variable (called \emph{extendable}). The  \emph{extendable} variable of a node is true if and only if the partial assignments that correspond to the paths from the root to it can be further extended to total assignments that respect the PBC the BDD represents.\\

The BDD can be built using a Breadth-First-Search starting in the root, shown as pseudocode in \ref{CodeBDDConstruction}. The version we use for encoding PBCs differs from the one introduced in \cite{Een2006TranslatingPC} because it is written to encode equations rather than inequations. Further, the used queue has additional functionalities: Given a node, it can check if it already contains a node with the same attribute values and can return said equal node. The \emph{updated sums} of successor nodes either are the same as their predecessors (\false{} was assigned) or are equal to the sum of their predecessor plus the weight value corresponding to the edge that led to them (\true{} was assigned).
\newpage
{
\pseudo{}
\renewcommand{\lstlistingname}{Algorithm}
\begin{lstlisting}[frame=single,caption={Pseudo Code of BDD construction},captionpos=b, label=CodeBDDConstruction, basicstyle=\footnotesize]
BuildBDD(PBC):
    queue = empty queue
    queue.append(createRoot())
    While not queue is empty:
        Node current = queue.first()
        If assigning next variable leads to a total assignment:
            Node cT = Create true successor node with updated sum
            Node cF = Create true successor node with updated sum
        Else:
            Node cT = Create true successor node with updated sum
            If cT.sum >= RHS:
                # true successor is a terminal node
            Else:
                # true successor is not a terminal node
                If cT not in queue:
                    queue.append(cT)
                Else:
                    cT = queue.get(cT)
            Node cF = Create false successor node with updated sum
            If cF.sum + sum of remaining weights < RHS:
                # false successor is a terminal node
            Else:
                # false successor is not a terminal node
                If cF not in queue:
                    queue.append(cF)
                Else:
                    cF = queue.get(cF)
        current.true_child = cT
        current.false_child = cF
    Return root
\end{lstlisting}
}
\newpage
Once the BDD is built, we can transform it into clauses. In \cite{Een2006TranslatingPC}, it is explained how this can be achieved by treating the BDD network as a circuit of ITEs (if-then-else gates). However, it suffices to know that the BDD can be transformed by doing a second Breadth-First-Search starting from the root and that for each visited node that is not a terminal node, the following six implications must be added to the set of formulas \cite{Een2006TranslatingPC}:

\begin{enumerate}
    \item If the \emph{extendable} variable of the current node is true and the variable corresponding to the leaving edges from this node is true, then it follows that the \emph{extendable} variable of the $true_{child}$ node is true.
    \item If the \emph{extendable} variable of the current node is true and the variable corresponding to the leaving edges from this node is false, then it follows that the \emph{extendable} variable of the $false_{child}$ node is true.
    \item If the \emph{extendable} variable of the current node is false and the variable corresponding to the leaving edges from this node is true, then it follows that the \emph{extendable} variable of the $true_{child}$ node is false.
    \item If the \emph{extendable} variable of the current node is false and the variable corresponding to the leaving edges from this node is false, then it follows that the \emph{extendable} variable of the $false_{child}$ node is false.
    \item If the \emph{extendable} variable of the $true_{child}$ node is true and the \emph{extendable} variable of the $false_{child}$ node is true, then it follows that the \emph{extendable} variable of the current node is true.
    \item If the \emph{extendable} variable of the $true_{child}$ node is false and the \emph{extendable} variable of the $false_{child}$ node is false, then it follows that \emph{extendable} variable of the current node is false.
\end{enumerate}

Additionally, we add a clause that only contains the positive literal corresponding to the \emph{extendable} variable of the root node. Also, when visiting a node during this second Breadth-First-Search, we only append its children that are not terminal nodes. For children that are terminal nodes, we add a clause to the set of clauses: 
\begin{itemize}
    \item If the child's sum is equal to the RHS, a clause containing a positive literal corresponding to the \emph{extendable} variable of the child is added.
    \item If the child's sum is unequal the RHS, a clause containing a negative literal corresponding to the \emph{extendable} variable of the child is added.
\end{itemize}
An example is shown in Figure \ref{fig:BDDExample} where the BDD and the clauses are depicted that are used to encode the PBC $6*v_1+4*v_2+2*v_3=6$. Sums are written inside the nodes, and \emph{extendable} variables are denoted as $a_i$ for $i\in \{1,2,...,10\}$. Edges with a circle correspond to assigning \false{}. Edges with a line correspond to assigning \true{}.

\begin{figure}
\centering
\includegraphics[width=\textwidth]{Figures/BDDExampleComposition3.png}
\caption{BDD and clauses to encode $6*v_1+4*v_2+2*v_3=6$}
\label{fig:BDDExample}
\end{figure}

\clearpage
\subsection{Encoding of PBCs using Adder Networks}\label{PBCEncodingAdderNetworks}
Adder Networks are built from Full and Half Adder nodes. A Full Adder (FA) is a node with three inputs and two outputs. A Half Adder (HA) is a node with two inputs and two outputs. The input and output values are binary (\true{}/\false{}). The outputs are name \emph{sum} and \emph{carry}, and their values are computed as follows. If at least two inputs are true, the carry output is true. Otherwise, it is false. So the carry value is already determined if two of three possible inputs are given. If one or three inputs are true, the sum output is true. Otherwise, it is false. This behaviour can be encoded into clauses, and as the name implies, it allows Adder Networks to compute sums over binary numbers. Binary numbers are Strings of bits, where each bit is either $1$ (\true) or $0$ (\false). The bit at position $k$ represents a value of $2^k$ and is called $k$-bit (This notation differs from the one used in \cite{Een2006TranslatingPC}). \\

An Adder Network used to sum numbers is explained best layer by layer. When adding binary numbers, the summand $k$-bits get fed into layer $k$. Layer $k$ then also takes the carry values from the previous layer $(k-1)$ as input for its nodes except if $k$ is $0$.  A layer's sum outputs are fed as input to adder nodes in the same layer (potentially requiring additional nodes) until only one input value remains. The remaining value of layer $k$ then corresponds to the value of the $k$-bit of the overall addition result. The carry outputs of a layer are ``carried over" to the next layer, where they become inputs to the adder nodes. The last layer has only one input, and its value gets directly treated as the output of this layer. Figure \ref{fig:AdderNetworkExample} shows an example network that can add three numbers, A, B and C, with values from $0$ to $7$ ($3$ bit long numbers).\\

\begin{figure}[ht!]
\centering
\includegraphics[width=0.72\textwidth]{Figures/AdderNetworkExample2.png}
\caption{Adder Network computing $6 + 4 + 2 = 12$}
\label{fig:AdderNetworkExample}
\end{figure}

As shown in \cite{Een2006TranslatingPC}, Adder Networks can be used to encode PBC. The weights of the PBC are taken as inputs for the network. Every input bit gets associated with the boolean variable that is multiplied by the weight they encode in the PBC. In practice, this is accomplished by feeding the boolean variable into the network directly (in the case of a \true{} bit) or by feeding nothing into the network (in the case of a \false{} bit). So if a boolean variable on the LHS of a PBC is \false{}, all the input bits of the corresponding weights are interpreted as $0$ by the network. The outputs of the adder nodes also get boolean variables assigned, including those outputs corresponding to the bit values of the overall result. These values should be equal to the corresponding bit values of the PBC's RHS as the PBCs we want to encode are equations. To enforce this, the values get compared, and corresponding unit clauses are added to the result set of clauses.\\

The encoding procedure (an adapted version of the algorithm  originally proposed in \cite{Een2006TranslatingPC}) is depicted as pseudocode in Figure \ref{CodeAdderNetwork}. What makes the algorithm presented in \cite{Een2006TranslatingPC} so efficient is that only the needed adder nodes are created, depending on how many \true{} inputs there are in a layer. This dynamic generation keeps the number of adder nodes that actually get transformed into clauses as small as possible. For example, to encode the PBC $6*v_1+4*v_2+2*v_3=6$, only two adder nodes (One Half Adder in Layer-1 and one Full Adder in Layer-2) must be transformed into clauses even though seven adders are needed to add up the numbers $6$, $4$ and $2$ (see Figure \ref{fig:AdderNetworkExample}).

{
\pseudo{}
\renewcommand{\lstlistingname}{Algorithm}
\begin{lstlisting}[frame=single,caption={Pseudocode of PBC encoding using an Adder Network},captionpos=b, label=CodeAdderNetwork, basicstyle=\footnotesize]
AdderNetworkEncoder(PBC):
    clauseSet = empty Set
    Translate PBC weights to binary
    Translate RHS to binary
    bucket = Create empty map, integer --> set
    For every PBC weight:
        For every k-bit of the weight:
            If the bit is true:
                If bucket has no key == k:
                    bucket[k] = new set
                Add variable that is multiplied with current weigth 
                in the PBC to bucket[k]
    networkOut = empty list
    k = 0
    While bucket has keys >= k:
        While bucket has no key == k:
            Append false-variable to networkOut
            k++
        While bucket[k] contains at least 3 variables:
            Create Full Adder with 3 variables of the k-bucket as input
            Add Full Adder clauses to clauseSet
            Create (k+1)-bucket if not present
            Put the Full Adders carry variable in bucket[k+1]
            Put the Full Adders sum variable in bucket[k]
        While bucket[k] contains at least 2 variables:
            Create Half Adder with 2 variables of the k-bucket as input
            Add Half Adder clauses to clauseSet
            Create (k+1)-bucket if not present
            Put the Half Adders carry variable in bucket[k+1]
            Put the Half Adders sum variable in bucket[k]
        Append the one remaining variable of bucket[k] to networkOut
        k++
    While networkOut.length < RHS.length:
        Append false-variable to networkOut
    While networkOut.length > RHS.length:
        Append false-variable to RHS
    For every RHS bit:
        If bit is true:
            Add a clause that contains a positive literal
            of the corresponding networkOut variable to clauseSet
        Else:
            Add a clause that contains a negative literal
            of the corresponding networkOut variable to clauseSet
    Return clauseSet
\end{lstlisting}
}

During algorithm \ref{CodeAdderNetwork}, Half and Full Adders are transformed into clauses, which can then be added to the result set of clauses that the procedure returns. Assuming the input variables of an adder are $x$, $y$ and $z$, the corresponding output variables are $s$ for the sum and $c$ for the carry. Then the needed implications (as presented in \cite{Een2006TranslatingPC} for Full Adders) to logically describe the adder nodes are as shown in Figure \ref{AdderNetworkClauses}.

\begin{figure}
    \centering
    %\def\arraystretch{1.5}
    \begin{tabular}{l l l}
    %\hline
    %&&\\
    \multicolumn {3}{l}{Full Adder clauses:}\\
     $\{~~x,~~y,~~z,\neg s\}$           & from & $(\neg x   \land \neg y    \land \neg z)   \rightarrow \neg s$ \\
     $\{~~x,\neg y,\neg z,\neg s\}$     & from & $(\neg x   \land ~~y       \land ~~z)      \rightarrow \neg s$ \\
     $\{\neg x,~~y,\neg z,\neg s\}$     & from & $(~~x      \land \neg y    \land ~~z)      \rightarrow \neg s$ \\
     $\{\neg x,\neg y,~~z,\neg s\}$     & from & $(~~x      \land ~~y       \land \neg z)   \rightarrow \neg s$ \\
     $\{\neg x,\neg y,\neg z, ~~s\}$    & from & $(~~x      \land ~~y       \land ~~z)      \rightarrow ~~s$ \\
     $\{\neg x, ~~y, ~~z, ~~s\}$        & from & $(~~x      \land \neg y    \land \neg z)   \rightarrow ~~s$ \\
     $\{~~x, \neg y, ~~z, ~~s\}$        & from & $(\neg x   \land ~~y       \land \neg z)   \rightarrow ~~s$ \\
     $\{~~x, ~~y, \neg z, ~~s\}$        & from & $(\neg x   \land \neg y    \land ~~z)      \rightarrow ~~s$ \\
     $\{\neg x, \neg y, ~~c\}$          & from & $(~~x      \land ~~y)      \rightarrow ~~c$ \\
     $\{\neg x, \neg z, ~~c\}$          & from & $(~~x      \land ~~z)      \rightarrow ~~c$ \\
     $\{\neg y, \neg z, ~~c\}$          & from & $(~~y      \land ~~z)      \rightarrow ~~c$ \\
     $\{~~x, ~~y, \neg c\}$          & from & $(\neg x      \land \neg y)   \rightarrow \neg c$ \\
     $\{~~x, ~~z, \neg c\}$          & from & $(\neg x      \land \neg z)   \rightarrow \neg c$ \\
     $\{~~y, ~~z, \neg c\}$          & from & $(\neg y      \land \neg z)   \rightarrow \neg c$ \\
     &&\\
     \multicolumn {3}{l}{Half Adder clauses:}\\
     $\{~~x, ~~y, \neg s\}$          & from & $(\neg x      \land \neg y)   \rightarrow \neg s$ \\
     $\{~~x, \neg y, ~~s\}$          & from & $(\neg x      \land ~~y)      \rightarrow ~~s$ \\
     $\{\neg x, ~~y, ~~s\}$          & from & $(~~x         \land \neg y)   \rightarrow ~~s$ \\
     $\{\neg x, \neg y, ~~s\}$       & from & $(~~x         \land ~~y)   \rightarrow \neg s$ \\
     $\{~~x, ~~y, \neg c\}$          & from & $(\neg x      \land \neg y)   \rightarrow \neg c$ \\
     $\{~~x, \neg y, \neg c\}$          & from & $(\neg x      \land ~~y)      \rightarrow \neg c$ \\
     $\{\neg x, ~~y, \neg c\}$          & from & $(~~x         \land \neg y)   \rightarrow \neg c$ \\
     $\{\neg x, \neg y, ~~c\}$       & from & $(~~x         \land ~~y)   \rightarrow ~~c$ \\
     %&&\\
     %\hline
    \end{tabular}
    \caption{Clauses for Full and Half Adders.\\
    (Inputs: $x$,$y$,$z$  Sum: $s$  Carry: $c$)}
    \label{AdderNetworkClauses}
\end{figure}

\newpage
\section{Normal Sudoku}
The normal Sudoku rules as introduced in \ref{NormalSudoku} can be broken down into the five constraints shown in Table \ref{constraints:NormalSudoku}, which can be encoded into clauses using the variable $s_{x,y,z}$, which is true iff cell $(x,y)$ has value $z$. The variable $s_{x,y,z}$ will be used further during the later shown formulae of other Sudoku Variants as the cell values play an important role in all of them. The encoding formulated in Table \ref{formulae:NormalSudoku} can be seen as a direct encoding using at-least-one and at-most-one clauses and was proposed by \cite{Lynce2006SudokuAsASATProblem} where it is called the minimal encoding.\\


\begin{table}[ht!]
    \centering
    \begin{tabular*}{\textwidth}{l @{\extracolsep{\fill}}  c  c}
        \hline
        \\
        Constraint & Formula & \#Clauses\\
        \\
        \hline
        \\
        At least one number from 1 to 9 appears in each grid cell. & (S-\ref{S-i}) & 81\\
        \\
        Every number appears at most once per row. & (S-\ref{S-ii}) & 2916\\
        \\
        Every number appears at most once per column. & (S-\ref{S-iii}) & 2916\\
        \\
        Every number appears at most once per box. & (S-\ref{S-iv}) and (S-\ref{S-v}) & 2916\\
        \\
        Every cell that contains a hint can only have that value. & (S-\ref{S-vi}) & 1/hint\\
        \\
        \hline
    \end{tabular*}
        \caption{Constraints of Normal Sudoku.}
    \label{constraints:NormalSudoku}
\end{table}
\begin{table}
    \centering
    \begin{tabular*}{\textwidth}{ l @{\extracolsep{\fill}} c}
    \hline
    \\
    $\displaystyle \bigwedge_{x=1}^9 \bigwedge_{y=1}^9 \bigvee_{z=1}^9 s_{x,y,z}$  & \consCount{S} \label{S-\roman{cons}}\\
    \\
    $\displaystyle \bigwedge_{y=1}^9 \bigwedge_{z=1}^9 \bigwedge_{x=1}^9 \bigwedge_{i=x+1}^9 \neg s_{x,y,z} \lor \neg s_{i,y,z}$  & \consCount{S} \label{S-\roman{cons}}\\
    \\
    $\displaystyle \bigwedge_{x=1}^9 \bigwedge_{z=1}^9 \bigwedge_{y=1}^9 \bigwedge_{i=y+1}^9 \neg s_{x,y,z} \lor \neg s_{x,i,z}$  & \consCount{S} \label{S-\roman{cons}}\\
    \\
    $\displaystyle \bigwedge_{z=1}^9 \bigwedge_{i=0}^2 \bigwedge_{j=0}^2 \bigwedge_{x=1}^3 \bigwedge_{y=1}^3 \bigwedge_{k=y+1}^3 \neg s_{(3*i+x),(3*j+y),z} \lor \neg s_{(3*i+x),(3*j+k),z}$  & \consCount{S} \label{S-\roman{cons}}\\
    \\
    $\displaystyle \bigwedge_{z=1}^9 \bigwedge_{i=0}^2 \bigwedge_{j=0}^2 \bigwedge_{x=1}^3 \bigwedge_{y=1}^3 \bigwedge_{k=x+1}^3 \bigwedge_{l=1}^3 \neg s_{(3*i+x),(3*j+y),z} \lor \neg s_{(3*i+k),(3*j+l),z}$  & \consCount{S} \label{S-\roman{cons}}\\
    \\
    $s_{x,y,input_{x,y}}$,  for every $(x,y)$ s.t. input$_{x,y}$ $\neq 0$  & \consCount{S} \label{S-\roman{cons}}\\
    \\
    \hline
    \end{tabular*}
    \caption{Formulae of clauses, Normal Sudoku.}
    \label{formulae:NormalSudoku}
\end{table}


    
\section{Anti-Knight}
To encode the Anti-Knight rule, one must ensure that for each grid cell (x,y), it is forbidden to have the same value as its Knight-Neighbours (cells that are a knights distance away). The one constraint needed to formulate the Anti-Knight rule is stated in Table \ref{constraints:AntiKnight}. The corresponding formula to encode it into clauses is shown in Table \ref{formulae:AntiKnight}. The set of Knight-Neighbours to cell (x,y) can be defined as:
\begin{center}
    $N(x,y) = \{(i,j)~|\textnormal{ cell }(i,j) \textnormal{ one knight distance from cell }(x,y)\}$.
\end{center}

\begin{table}[ht!]
    \centering
    \begin{tabular*}{\textwidth}{l @{\extracolsep{\fill}} c  c}
        \hline
        \\
        Constraint & Formula & \#Clauses\\
        \\
        \hline
        \\
        \makecell[cl]{Cells that are one knight distance apart (neighbours) \\ must have different values.} & (AK-\ref{AK-i}) & 2016\\
        \\
        \hline
    \end{tabular*}
        \caption{Constraints of Anti-Knight rule.}
    \label{constraints:AntiKnight}
\end{table}

\begin{table}[ht!]
    \centering
    \begin{tabular*}{\textwidth}{ l @{\extracolsep{\fill}} c}
    \hline
    \\
    $\displaystyle \bigwedge_{x=1}^9 \bigwedge_{y=1}^9 \bigwedge_{(i,j)\in N(x,y)} \bigwedge_{z=1}^9 \neg s_{x,y,z} \lor \neg s_{i,j,z}$ &\consCount{AK} \label{AK-\roman{cons}}\\\
    \\
    \hline
    \end{tabular*}
    \caption{Formulae of clause, Anti-Knight rule.}
    \label{formulae:AntiKnight}
\end{table}


\newpage
\section{Killer Sudoku}\label{encoding:killer}
\paragraph{Using PBCs:} For every killer cage of the input, we create a list of its cells. From every list, a PBC is created as follows: For every cell (x,y) of a cage, we add $\sum_{i=1}^{9} (s_{x,y,i}*i)$ to the left-hand side of the PBC. The right-hand side of the PBC is set to the target sum that was given as input. The different PBCs (one for every killer cage) can then be encoded into clauses, as explained in \ref{PBCEncodingBDD} and \ref{PBCEncodingAdderNetworks}.

\paragraph{Using PBCs + Combinations:}
The PBC approach can be further optimized, because, given a fixed number of summands, not all values from 1 to 9 can be used to achieve a particular sum. For example, if a cage has a target sum of 8 and consists of three cells, the number of possible value combinations to achieve the target sum is fairly limited. There are only two possible value combinations $1+2+5=8$ and $1+3+4=8$, so the allowed values the cells could take are $1$, $2$, $3$, $4$ and $5$. When constructing the PBC, this knowledge can be used to reduce the number of variable-value products on the left-hand side of the equation. For every cell in a cage, we only add the variable times the corresponding value (to the left-hand side) if the value is an allowed one.

\paragraph{Using Combinations:}
Another possibility is to completely abandon PBCs and exploit that only certain value combinations are possible given a cage with a fixed target sum and fixed number of cells that belong to it. To encode this every combination is given a corresponding variable $s_{v}$, for $v \in V_g  \subset\mathbb{N}$, which is true iff the corresponding combination is used in a certain cage. The set of all cages is annotated as $G$, and the set of all possible combinations to achieve the target sum of a cage $g \in G$ is denoted as $C_g$. The constraints needed to encode the Killer Sudoku rules without the use of PBCs are stated in Table \ref{constraints:Killer}, further the formulae that encode these constraint to clauses are described in Table \ref{formulae:KillerSudoku}.

\begin{table}[ht!]
    \centering
    \begin{tabular*}{\textwidth}{l @{\extracolsep{\fill}} c}
        \hline
        \\
        Constraint & Formula\\
        \\
        \hline
        \\
        \makecell[cl]{For every cage $g \in G$ and possible combination $c \in C_g$ it\\
        holds that, either the cage's target sum is not achieved using\\
        combination $c$ or every cage cell contains at least one value\\
        of the combination $c$.} & (K-\ref{K-i})\\
        \\
        In every cage $g \in G$ at least one combination $c \in C_g$ is used. & (K-\ref{K-ii})\\
        \\
        In every cage $g \in G$ at most one combination $c \in C_g$ is used. & (K-\ref{K-iii})\\
        \\
        \makecell[cl]{Every value from 1 to 9 appears at most once within the\\
        cells of a cage $g \in G$.} & (K-\ref{K-iv})\\
        \\
        \hline
    \end{tabular*}
        \caption{Constraints of Killer Sudoku rules.}
    \label{constraints:Killer}
\end{table}

\begin{table}
    \centering
    \begin{tabular*}{\textwidth}{ m{70mm} l @{\extracolsep{\fill}} c}
    \hline
    \\
    $\displaystyle \bigwedge_{g:G} \bigwedge_{c:C_g} \bigwedge_{(x,y):g} -s_v \lor \bigvee_{z:c}  s_{x,y,z}$ & & \consCount{K} \label{K-\roman{cons}}\\
    \\
    $\displaystyle \bigwedge_{g:G} \bigvee_{v:V_g} s_{v}$ & & \consCount{K} \label{K-\roman{cons}}\\
    \\
    $\displaystyle \bigwedge_{g:G} \bigwedge_{v':V_g} \bigwedge_{v'':V_g} \neg s_{v'} \lor \neg s_{v''}$  & with $v' < v''$ & \consCount{K} \label{K-\roman{cons}}\\
    \\
    $\displaystyle \bigwedge_{g:G} \bigwedge_{(x_i,y_i):g} \bigwedge_{(x_j,y_j):g} \bigwedge_{z=1}^{9} \neg x_i y_i z \lor \neg x_j y_j z$ & with $(x_i,y_i) \neq (x_j,y_j)$ &\consCount{K} \label{K-\roman{cons}}\\
        \\
    \hline
    \end{tabular*}
    \caption{Formulae of clauses, Killer Sudoku rules.}
    \label{formulae:KillerSudoku}
\end{table}

\clearpage

\section{Arrowheads}\label{Encoding:Arrowheads}
Arrowheads demand a total or partial order between two cells. The sets of all tuples of two cells under total or partial order are respectively noted as $TO$ and $PO$. The first cell $(x_1,y_1)$ of such a tuple  $((x_1,y_1), (x_2,y_2)) \in TO \cup PO$ shall have a smaller (or equal) value than the second one $(x_2,y_2)$. To enforce this, the constraints shown in \ref{tab:ArrowHeads} are encoded into clauses using the support encoding (see Table \ref{formulae:Arrowhead}). In formula AH-\ref{AH-i}, the values of $z_1$ go from $1$ to $9$, which ensures that the first cell of a tuple can not have a value of 9. In formula AH-\ref{AH-ii}, on the other hand, $z_1$ iterates from $2$ to $9$ because if the first cell has a value of $1$, all values from $1$ to $9$ are allowed for the second cell.\\

\begin{table}[ht!]
    \centering
    \begin{tabular*}{\textwidth}{l @{\extracolsep{\fill}} c}
        \hline
        \\
        Constraint & Formula\\
        \\
        \hline
        \\
        \makecell[cl]{For every $((x_1,y_1), (x_2,y_2)) \in TO$, either cell $(x_1,y_1)$ has not\\
        value $z_1\in \{1,..,9\}$ or cell $(x_2,y_2)$ has value $z_2 > z_1, z_2 \in \{2,..,9\}$.} & (AH-\ref{AH-i})\\
        \\
        \makecell[cl]{For every $((x_1,y_1), (x_2,y_2)) \in PO$, either cell $(x_1,y_1)$ has not\\
        value $z_1\in \{2,..,9\}$ or cell $(x_2,y_2)$ has value $z_2 \geq z_1, z_2 \in \{1,..,9\}$.} & (AH-\ref{AH-ii})\\
        \\
        \hline
    \end{tabular*}
        \caption{Constraints of Arrowhead rules.}
    \label{tab:ArrowHeads}
\end{table}
\begin{table}[h!]
    \centering
    \begin{tabular*}{\textwidth}{ l l @{\extracolsep{\fill}} c}
    \hline
    \\
    $\displaystyle \bigwedge_{((x_1,y_1),(x_2,y_2)):TO} \bigwedge_{z_1=1}^{9} \neg s_{x_1,y_1,z_1} \lor \bigvee_{z_2=z_1+1}^{9} s_{x_2,y_2,z_2}$ & & \consCount{AH} \label{AH-\roman{cons}}\\
    \\
    $\displaystyle \bigwedge_{((x_1,y_1),(x_2,y_2)):PO} \bigwedge_{z_1=2}^{9} \neg s_{x_1,y_1,z_1} \lor \bigvee_{z_2=z_1}^{9} s_{x_2,y_2,z_2}$ & & \consCount{AH} \label{AH-\roman{cons}}\\
        \\
    \hline
    \end{tabular*}
    \caption{Formulae of clauses, Arrowhead rules.}
    \label{formulae:Arrowhead}
\end{table}

\clearpage

\section{Thermometers (Hidden)}
As the Thermometer rule also demands a total (or partial) order between the cells of the thermometers, one can reuse the rules and formulas shown for Arrowheads in \ref{Encoding:Arrowheads}. Starting from a thermometers bulb, one can conceptually place arrowheads between every cell pair along the thermometer, enforcing a total (or partial) order between all cells.\\

Further constraints and variables are needed for the more complicated puzzle, where the solver must also deduce the thermometer positions. When encoding this puzzle, we distinguish two main cases: if the entire thermometer number $t$ fits inside the $9\times9$ grid when its bulb is placed at $(x,y)$, the clauses of TH-\ref{TH-iv}, TH-\ref{TH-v} and TH-\ref{TH-vi} are added. Otherwise the clauses of TH-\ref{TH-vii} are added. The corresponding constraints are defined in Table \ref{Constraints:Thermometers}, and the formulae in Table \ref{formulae:thermometers} show how to encode them into clauses. Details on the used notation and variables within the formulae can be found in Table \ref{notation:ThermometersHidden}.\\


\begin{table}[!ht]
    \centering
    \begin{tabular*}{\textwidth}{l c l}
    \hline
    \\
    Notation and Variables: &&\\
    \\
    \hline
    \\
    $T$                     &= &\begin{tabular}{l c l}
                                    \multicolumn{3}{l}{List of given thermometers}\\
                                    \end{tabular}\\
    \\
    $T.size$                &= &\begin{tabular}{l c l}
                                    \multicolumn{3}{l}{Number of given thermometers}\\
                                    \end{tabular}\\
    \\
    $T[t]$                  &= &\begin{tabular}{l c l}
                                    \multicolumn{3}{l}{Thermometer number t in list of all thermometers}\\
                                    \end{tabular}\\
    \\
    $T[t].size$             &= &\begin{tabular}{l c l}
                                    \multicolumn{3}{l}{Number of cells in thermometer number $t$}\\
                                    \end{tabular}\\
    \\
    $\rchi(t,x,y,d)$        &= &\begin{tabular}{l c l}
                                    \multicolumn{3}{l}{$x$-coordinate of the cell that is at depth $d$ of the}\\
                                \end{tabular}\\
                            &  &\begin{tabular}{l c l}
                                \multicolumn{3}{l}{thermometer number $t$ if its bulb is placed in cell $(x,y)$}\\
                            \end{tabular}\\
    \\
    $\Upsilon(t,x,y,d)$     &= &\begin{tabular}{l c l}
                                    \multicolumn{3}{l}{$y$-coordinate of the cell that is at depth $d$ of the}\\
                                \end{tabular}\\
                            &  &\begin{tabular}{l c l}
                                \multicolumn{3}{l}{thermometer number $t$ if its bulb is placed in cell $(x,y)$}\\
                            \end{tabular}\\
    \\
    $Arrowhead(t,x,y,d)$    &= &\begin{tabular}{l c l}
                                    \multicolumn{3}{l}{Set of clauses needed to encode that the cell at depth}\\
                                \end{tabular}\\
                            &  &\begin{tabular}{l c l}
                                \multicolumn{3}{l}{$d$ of thermometer number $t$ is smaller than the cell at}\\
                                \multicolumn{3}{l}{depth $d+1$, given the thermometer has its bulb in cell}\\
                                \multicolumn{3}{l}{$(x,y)$.}\\
                            \end{tabular}\\
    \\
    $s_{t,d,x,y}$           &: &\begin{tabular}{l c l}
                                    \multicolumn{3}{l}{Is true if and only if the cell at depth $d$ of thermometer}\\
                                \end{tabular}\\
                            &  &\begin{tabular}{l c l}
                                \multicolumn{3}{l}{number $t$ is located in grid cell $(x,y).$}\\
                                 t          &$\in$  &$\{1,..,81\}$\\
                                 d,x,y      &$\in$  &$\{1,..,9\}$\\
                                 Digits     &:      &ttdxy\\
                                 Range:     &:      &$01111$ to $81999$\\
                                 \multicolumn{3}{l}{\makecell[cl]{In practice $t$ has an offset of +10 to avoid the leading 0.\\
                                 So the actual variable range is: $11111$ to $91999$.}}\\
                            \end{tabular}\\
    \\
    \hline
\end{tabular*}
    \caption{Notations and variables, Thermometers-Hidden rules.}
    \label{notation:ThermometersHidden}
\end{table}





\begin{table}[ht!]
    \centering
    \begin{tabular*}{\textwidth}{l @{\extracolsep{\fill}} c}
        \hline
        \\
        Constraint & Formula\\
        \\
        \hline
        \\
        \makecell[cl]{Every cell of every thermometer is located in at least one\\
        grid cell.} & (TH-\ref{TH-i})\\
        \\
        \makecell[cl]{At most one thermometer cell is located in one grid cell.} & (TH-\ref{TH-ii}) and (TH-\ref{TH-iii})\\
        \\
        \makecell[cl]{If the cell at depth $d$ of thermometer number $t$ is located\\
        in grid cell $(x, y)$, then the thermometer cell at depth\\
        $d+1$ must be located next to it (corresponding to the\\
        thermometer's shape), except if $d$ = T[t].size.} & (TH-\ref{TH-iv})\\
        \\
        \makecell[cl]{If the cell at depth $d$ of thermometer number $t$ is located\\
        in grid cell $(x, y)$, then the thermometer cell at depth\\
        $d-1$ must be located next to it (corresponding to the\\
        thermometer's shape), except if $d$ = 0.} & (TH-\ref{TH-v})\\
        \\
        \makecell[cl]{If the cell at depth $d$ of thermometer number $t$ is located\\
        in grid cell $(x, y)$, then the cell value of $(x,y)$ must be\\
        lower than that of the thermometer cell at depth $d+1$.} & (TH-\ref{TH-vi})\\
        \\
        \makecell[cl]{If not the entire thermometer with number t is inside\\
        the $9 \times 9$ when its bulb is placed at cell $(x,y)$, then the\\
        cell at depth $d$ of thermometer number $t$ can not be\\
        located at the cell it covers in this situation.} & (TH-\ref{TH-vii})\\
        \hline
    \end{tabular*}
        \caption{Constraints of the Thermometers-Hidden rules.}
    \label{Constraints:Thermometers}
\end{table}
\begin{table}[ht!]
    \begin{tabular*}{\textwidth}{ l l @{\extracolsep{\fill}} c}
    \hline
    \\
    $\displaystyle \bigwedge_{t=1}^{T.size} \bigwedge_{d=1}^{T[t].size} \bigvee_{x=1}^{9} \bigvee_{y=1}^{9} s_{t,d,x,y}$ & & \consCount{TH} \label{TH-\roman{cons}}
    \\
    \\
    $\displaystyle \bigwedge_{t=1}^{T.size} \bigwedge_{d=1}^{T[t].size} \bigwedge_{k=d+1}^{T[t].size} \bigwedge_{x=1}^{9} \bigwedge_{y=1}^{9}  \neg s_{t,d,x,y} \lor \neg s_{t,k,x,y}$ & & \consCount{TH} \label{TH-\roman{cons}}
    \\
    \\
    $\displaystyle \bigwedge_{t=1}^{T.size} \bigwedge_{k=t+1}^{T.size} \bigwedge_{d=1}^{T[t].size}  \bigwedge_{l=1}^{T[t].size} \bigwedge_{x=1}^{9} \bigwedge_{y=1}^{9} \neg s_{t,d,x,y} \lor \neg s_{k,l,x,y}$ & & \consCount{TH} \label{TH-\roman{cons}}
    \\
    \\
    $\displaystyle \bigwedge_{t=1}^{T.size} \bigwedge_{x=1}^{9} \bigwedge_{y=1}^{9} \bigwedge_{d=1}^{T[t].size-1} \neg s_{t,d,\rchi(t,x,y,d), \Upsilon(t,x,y,d)} \lor s_{t,d+1,\rchi(t,x,y,d+1), \Upsilon(t,x,y,d+1)}$ & & \consCount{TH} \label{TH-\roman{cons}}
    \\
    \\
    $\displaystyle \bigwedge_{t=1}^{T.size} \bigwedge_{x=1}^{9} \bigwedge_{y=1}^{9} \bigwedge_{d=1}^{T[t].size-1} s_{t,d,\rchi(t,x,y,d), \Upsilon(t,x,y,d)} \lor \neg s_{t,d+1,\rchi(t,x,y,d+1), \Upsilon(t,x,y,d+1)}$ & & \consCount{TH} \label{TH-\roman{cons}}
    \\
    \\
    $\displaystyle \bigwedge_{t=1}^{T.size} \bigwedge_{x=1}^{9} \bigwedge_{y=1}^{9} \bigwedge_{d=1}^{T[t].size-1} \neg s_{t,d,\rchi(t,x,y,d), \Upsilon(t,x,y,d)} \lor Arrowhead(t,x,y,d)$ & & \consCount{TH} \label{TH-\roman{cons}}
    \\
    \\
    $\displaystyle \bigwedge_{t=1}^{T.size} \bigwedge_{x=1}^{9} \bigwedge_{y=1}^{9} \bigwedge_{d=1}^{T[t].size} \neg s_{t,d,\rchi(t,x,y,d), \Upsilon(t,x,y,d)}$ ~~~~~~\textnormal{with} $\rchi(\cdot), \Upsilon(\cdot) \in [1,9]$ & & \consCount{TH} \label{TH-\roman{cons}}\\
        \\
    \hline
\end{tabular*}
    \caption{Formulae of clauses, Thermometers-Hidden rules.}
    \label{formulae:thermometers}
\end{table}

\FloatBarrier
\newpage
\section{Sandwich Sum}
Sandwich Sums can be given for rows and columns. We will elaborate on the row constraints from which the formulas for the column constraints can be derived by swapping $x$ and $y$, respectively. Similar to the Killer Sudoku rules, multiple optimizations can be (incrementally) made when encoding the Sandwich Sum rules.

\paragraph{Using PBCs:} Assuming that all lengths of sandwiches are possible to achieve a certain sum. We must encode all possible sandwich lengths ($0$ to $7$) and all corresponding positions for the cells with values $1$ and $9$. Also, we encode a PBC for every possible sandwich length and position. The LHS of said PBC is $\sum_{i=1}^{9} (s_{x,y,i}*i)$, and the RHS is set to the current rows sandwich sum. Since the sandwich can only be at one position in a row at once, only one of all the PBC will be true. To ensure that a PBC's clauses are satisfied if the sandwich is not in its corresponding position, we add the negative literal of the corresponding $s_v$ to each clause (see formulae SW-\ref{SW-vii}, SW-\ref{SW-viii} and SW-\ref{SW-ix} in Table \ref{formulae:SandwichSum}). To encode the Sandwich Sum rules for rows without the incorporation of further knowledge, the following constraints/formulas are used: SW-\ref{SW-i}, SW-\ref{SW-ii}, SW-\ref{SW-vii}, SW-\ref{SW-x} and SW-\ref{SW-xi}.

\paragraph{Using PBCs + Combinations for Lengths:} Given the sandwich sum of a row, one can already make statements about the number of cells involved in the sum. For example, given a sandwich sum of $8$, one can conclude that the sandwich has either length $3$, $2$, or $1$ (contains $3$, $2$ or $1$ cells, not counting the border cells with values $1$ and $9$). This already reduces the number of combinations of sandwich lengths and positions and can be utilized to encode the rules using the constraints/formulas: SW-\ref{SW-iii}, SW-\ref{SW-iv}, SW-\ref{SW-v}, SW-\ref{SW-vi}, SW-\ref{SW-viii}, SW-\ref{SW-x} and SW-\ref{SW-xi}.

\paragraph{Using Combinations for PBCs + Combinations for Lengths:} Given the sandwich sum of a row, one can not only make statements about the number of cells involved in the sum but also about the possible values of these cells given a certain sandwich length (Similar to the possible cell values in a killer cage given the cages sum and size). This reduces the number of summands of the LHS of PBCs and can be combined with the knowledge about possible sandwich lengths. To encode the Sandwich Sum rules using this additional knowledge, the following constraints/formulas: are used: SW-\ref{SW-iii}, SW-\ref{SW-iv}, SW-\ref{SW-v}, SW-\ref{SW-vi}, SW-\ref{SW-ix}, SW-\ref{SW-x}, SW-\ref{SW-xi} and SW-\ref{SW-xii}.\\
\\
The different constraints and formulae used for these three encoding versions are listed in Tables \ref{Constraints:SandwichSum} and \ref{formulae:SandwichSum}, respectively. Notation and variables used in the formulae are further explained in Table \ref{notation:SandwichSum}.

\begin{table}
    \centering
    \begin{tabular*}{\textwidth}{l @{\extracolsep{\fill}} c}
        \hline
        \\
        Constraint on rows & Formula\\
        \\
        \hline
        \\
        \makecell[cl]{If the sandwich of a row is at a certain position, the cells\\
        with values 1 and 9 must be positioned at its left and right\\
        end.} & (SW-\ref{SW-i}) and (SW-\ref{SW-ii})\\
        \\
        \makecell[cl]{If the sandwich of a row (with compatible length regarding\\
        the sandwich sum) is at a certain position, the cells with\\
        values 1 and 9 must be positioned at its left and right end.} & (SW-\ref{SW-iii}) and (SW-\ref{SW-iv})\\
        \\
        \makecell[cl]{The cells with values 1 and 9 can not have a certain\\
        number of cells between them, if the values from these\\
        number of cells can not add up to the rows sandwich sum.} & (SW-\ref{SW-v}) and SW-\ref{SW-vi})\\
        \\
        \makecell[cl]{If the sandwich of a row is at a certain position, then the\\
        corresponding cells must add up to the row's sandwich\\
        sum.} & (SW-\ref{SW-vii})\\
        \\
        \makecell[cl]{If the sandwich of a row (with compatible length regarding\\
        the sandwich sum) is at a certain position, then the\\
        corresponding cells must add up to the row's sandwich\\
        sum.} & (SW-\ref{SW-viii})\\
        \\
        \makecell[cl]{If the sandwich of a row (with compatible length regarding\\
        the sandwich sum) is at a certain position, then the\\
        corresponding cells must add up to the row's sandwich sum\\
        using only compatible values regarding the sandwich sum.} & (SW-\ref{SW-ix})\\
        \\
        \makecell[cl]{The sandwich must be in at least one position.} & (SW-\ref{SW-x})\\
        \\
        \makecell[cl]{The sandwich must be in at most one position.} & (SW-\ref{SW-xi})\\
        \\
        \makecell[cl]{If the sandwich of a row is at a certain position, the cell\\
        values of the cells that are added up to the sandwich sum\\
        can not be incompatible regarding the sandwich sum.} & (SW-\ref{SW-xii})\\
        \\
        \hline
    \end{tabular*}
        \caption{Constraints of Sandwich Sum rules.}
    \label{Constraints:SandwichSum}
\end{table}


\begin{table}
    \centering
    \begin{tabular*}{\textwidth}{l c l}
    \hline
    \\
    Notation and Variables: &&\\
    \\
    \hline
    \\
    $PBC(x_1,x_2,y,sum)$    &= &Set of clauses needed to encode that the cells between\\
                            &  &$(x_1,y)$ and $(x_2,y)$ have values $\in \{1,...,9\}$ that add up\\
                            &  &to $sum$.\\
    \\
    $PBC(x_1,x_2,y,sum,Val)$&= &Set of clauses needed to encode that the cells between\\
                            &  &$(x_1,y)$ and $(x_2,y)$ have values $\in \textit{Val}$ that add up to\\
                            &  &$sum$.\\
    \\
    $L(y)$                  &= &Set of numbers corresponding to the sizes of cell sets\\
                            &  &for which it is possible to achieve the sandwich sum of\\
                            &  &row $y$, given that each cell can only have values from\\
                            &  &1 to 9.\\
                            &  &$L(y)\subseteq \{0,1,...,7\}$
    \\
    $\bar{L}(y)$            &= &$\{0,...,7\}\setminus L(y)$\\
    \\
    $Z(y,l)$                &= &Set of integer values for which it is possible to achieve\\
                            &  &the sandwich sum of row $y$ if $l$ different one of them\\
                            &  &are added.\\
                            &  &$Z(y,l)\subseteq \{1,...,9\}$
    \\
    $\bar{Z}(y,l)$          &= &$\{1,...,9\}\setminus Z(y,l)$\\
    \\
    $\mathcal{S}(x,y,l)$    &= &Dynamically assigned variable $s_v$ for each combination\\
                            &  &of $x$, $y$ and $l$ that is true if the sandwich of row $y$ has\\
                            &  &its borders in cell $(x,y)$ and cell $(x+1+l,y)$. \\
                            &  &$v \in V_y \subset \mathbb{N}$\\
    \\
    \hline
    \end{tabular*}
    \caption{Notations and variables, Sandwich Sum rules.}
    \label{notation:SandwichSum}
\end{table}




\begin{table}[ht!]
    \begin{tabular*}{\textwidth}{ l l @{\extracolsep{\fill}} c}
    \hline
    \\
    $\displaystyle \bigwedge_{y=1}^{9} \bigwedge_{l=0}^{7} \bigwedge_{x=1}^{9}  \neg s_{x,y,1} \lor \neg s_{(x+l+1),y,9} \lor \mathcal{S}(x,y,l)$ & & \consCount{SW} \label{SW-\roman{cons}}\\
    \\
    $\displaystyle \bigwedge_{y=1}^{9} \bigwedge_{l=0}^{7} \bigwedge_{x=1}^{9}  \neg s_{x,y,9} \lor \neg s_{(x+l+1),y,1} \lor \mathcal{S}(x,y,l)$ & & \consCount{SW} \label{SW-\roman{cons}}\\
    \\
    $\displaystyle \bigwedge_{y=1}^{9} \bigwedge_{l:L(y)} \bigwedge_{x=1}^{9}  \neg s_{x,y,1} \lor \neg s_{(x+l+1),y,9} \lor \mathcal{S}(x,y,l)$ & & \consCount{SW} \label{SW-\roman{cons}}\\
    \\
    $\displaystyle \bigwedge_{y=1}^{9} \bigwedge_{l:L(y)} \bigwedge_{x=1}^{9}  \neg s_{x,y,9} \lor \neg s_{(x+l+1),y,1} \lor \mathcal{S}(x,y,l)$ & & \consCount{SW} \label{SW-\roman{cons}}\\
    \\
    $\displaystyle \bigwedge_{y=1}^{9} \bigwedge_{l:\bar{L}(y)} \bigwedge_{x=1}^{9-l-1}  \neg s_{x,y,1} \lor \neg s_{(x+l+1),y,9}$ & & \consCount{SW} \label{SW-\roman{cons}}\\
    \\
    $\displaystyle \bigwedge_{y=1}^{9} \bigwedge_{l:\bar{L}(y)} \bigwedge_{x=1}^{9-l-1}  \neg s_{x,y,9} \lor \neg s_{(x+l+1),y,1}$ & & \consCount{SW} \label{SW-\roman{cons}}\\
    \\
    $\displaystyle \bigwedge_{y=1}^{9} \bigwedge_{l=0}^{7} \bigwedge_{x=1}^{9} \bigwedge_{\varphi \in PBC(x,x+l+1,y,sum)} \varphi \lor \neg \mathcal{S}(x,y,l)$ & & \consCount{SW} \label{SW-\roman{cons}}\\
    \\
    $\displaystyle \bigwedge_{y=1}^{9} \bigwedge_{l:L(y)} \bigwedge_{x=1}^{9}  \bigwedge_{\varphi \in PBC(x,x+l+1,y,sum)} \varphi \lor \neg \mathcal{S}(x,y,l)$ & & \consCount{SW} \label{SW-\roman{cons}}\\
    \\
    $\displaystyle \bigwedge_{y=1}^{9} \bigwedge_{l:L(y)} \bigwedge_{x=1}^{9}  \bigwedge_{\varphi \in PBC(x,x+l+1,y,sum, Z(y,l))} \varphi \lor \neg \mathcal{S}(x,y,l)$ & & \consCount{SW} \label{SW-\roman{cons}}\\
    \\
    $\displaystyle \bigwedge_{y=1}^{9} \bigvee_{v:V_y} s_v$ & & \consCount{SW} \label{SW-\roman{cons}}\\
    \\
    $\displaystyle \bigwedge_{y=1}^{9} \bigwedge_{v':V_y} \bigwedge_{v'':V_y} \neg s_{v'} \lor \neg s_{v''}$ & with $v' < v''$ & \consCount{SW} \label{SW-\roman{cons}}\\
    \\
    $\displaystyle \bigwedge_{y=1}^{9} \bigwedge_{l:L(y)} \bigwedge_{x=1}^{9}  \bigwedge_{z:\bar{Z}(y,l)} \bigwedge_{x'=x+1}^{x+l} \neg s_{x',y,z} \lor \neg \mathcal{S}(x,y,l)$ & & \consCount{SW} \label{SW-\roman{cons}}\\
    \\
    \hline
\end{tabular*}
    \caption{Formulae of clauses, Sandwich Sum rules.}
    \label{formulae:SandwichSum}
\end{table}

\FloatBarrier
\newpage
\section{Secret Direction}
To encode the secret direction rules, we introduce a variable that is true iff a cell $(x,y)$ is part of the hidden path at a certain depth $d$. We define the initial cell to have a depth of $0$, and its successors in the path have depths $1$, $2$, et cetera. The depth can be used to ensure that every cell can only be in the path once, those prohibiting loops. That loops can not be part of the solution path can be inferred from the definition of the next step in a path, which is always determined by the position of the value $9$ in the current $(3\times3)$-box and the value of the current path cell, making it impossible that one cell of the path has two different successors. So loops cannot be part of the solution path because it would not be possible to break out of them.\\
\\
From the fact that the position of the value $9$ is needed to determine the direction in which the successor of a cell in the path lays, we can also derive that if the center of a $(3\times3)$-box has value $9$, all other cells of that box cannot be in the path, because for them no successor direction would be defined.\\
\\
Since the Sudoku grid has only a size of $9\times9$ and the cell value of a cell in the path determines the distance to its successor, we can conclude that cells with a value of $9$ can not be part of the path (except as final cell) because after them the path would step outside the grid.\\
\\
Combining the previous two deductions, we can compute a sufficiently low upper bound for the maximal path length of $81-8-8 = 65$. Given these analyses, the constraints shown in \ref{Constraints:SecretDirection} can be formulated and encoded into clauses as shown in \ref{formulae:SecretDirection}. Details about the used notation and helping functions can be found in \ref{notation:SecretDirection}.



\begin{table}
    \centering
    \begin{tabular*}{\textwidth}{l @{\extracolsep{\fill}} c}
        \hline
        \\
        Constraint & Formula\\
        \\
        \hline
        \\
        \makecell[cl]{At least one $(3\times 3)$-box has a cell with value 9 as center.} & (SD-\ref{SD-i})\\
        \\
        \makecell[cl]{At most $(3\times 3)$-box has a cell with value 9 as center.} & (SD-\ref{SD-ii}) and (SD-\ref{SD-iii})\\
        \\
        \makecell[cl]{A $(3\times 3)$-box center cell that has value 9 is part of the\\
        path in some depth.} & (SD-\ref{SD-iv})\\
        \\
        \makecell[cl]{A cell can be at most once in the path (can only be in\\
        at most one depth).} & (SD-\ref{SD-v})\\
        \\
        \makecell[cl]{The path can have at most one cell in every depth.} & (SD-\ref{SD-vi}) and (SD-\ref{SD-vii})\\
        \\
        \makecell[cl]{Cells that are in the path, and are center cells of a\\
        $(3\times 3)$-box, have not the value 9. (All cells with value 9,\\
        that are in the path are center cells.)} & (SD-\ref{SD-viii})\\
        \\
        \makecell[cl]{The non center cells of the $(3\times 3)$-box with a center cell\\
        that has value 9, are not in the path.} & (SD-\ref{SD-vi}) and (SD-\ref{SD-ix})\\
        \\
        \makecell[cl]{If a cell $(x,y)$ is not the center cell of a $(3\times 3)$-box, has\\
        value $z$, is in the path at depth $d$, and the the position of\\
        the $9$-valued cell in the same $(3\times3)$-box hints in a certain\\
        direction. Then the cell $(x_s,y_s)$ that is $z$ steps in certain\\
        direction away from $(x,y)$ is also in the path, at layer\\
        $(d+1)$.} & (SD-\ref{SD-xi})\\
        \\
        \hline
    \end{tabular*}
        \caption{Constraints of Secret Direction rules.}
    \label{Constraints:SecretDirection}
\end{table}

\begin{table}
    \centering
    \begin{tabular*}{\textwidth}{m{40mm} m{1mm} l}
    \hline
    \\
    \multicolumn{3}{l}{Notation and Variables:}\\
    \\
    \hline
    \\
    $M$             &= &\{(2,2), (2,5), (2,8), (5,2), (5,5), (5,8), (8,2), (8,5), (8,8)\}\\
    \\
    $\Psi$          &= &\{(1,1), (1,2), (1,3), (2,1), (2,3), (3,1), (3,2), (3,3)\}\\
    \\
    $\mathcal{G}$   &= & $\{(a,b)~|~a,b\in \{1,2,3,4,5,6,7,8,9\}\}$\\
    \\
    $\mathcal{F}_1(x_b,y_b,x',y',d,x_n,y_n,z)$   &= &$\begin{cases}
    \makecell[cl]{(\neg s_{1,(3*x_b+x'),(3*y_b+y'),d}\\
    ~~\lor \neg s_{(3*x_b+x_n),(3*y_b+y_n),9}\\
    ~~\lor \neg s_{x,y,z})} & (x_s,y_s) \notin \mathcal{G}\\
    \makecell[cl]{(\neg s_{1,(3*x_b+x'),(3*y_b+y'),d}\\
    ~~\lor \neg s_{(3*x_b+x_n),(3*y_b+y_n),9}\\
    ~~\lor \neg s_{x,y,z}\\
    ~~\lor s_{1,succ_x(x_b,x',x_n,z),succ_y(y_b,y',y_n,z),(d+1)})} & (x_s,y_s) \in \mathcal{G}
    \end{cases}$\\
    \\
    $succ_x(x_b,x',x_n,z)$  &= &$\begin{cases}
                                    (3*x_b+x')-z    & x_n = 1 \\
                                    (3*x_b+x')      & x_n = 2 \\
                                    (3*x_b+x')+z    & x_n = 3 \\
                                    \end{cases}$\\
    \\
    $succ_y(y_b,y',y_n,z)$  &= &$\begin{cases}
                                    (3*y_b+y')-z    & y_n = 1 \\
                                    (3*y_b+y')      & y_n = 2 \\
                                    (3*y_b+y')+z    & y_n = 3 \\
                                    \end{cases}$\\
    \\
    $s_{1,x,y,d}$       &:  &\begin{tabular}{l c l}
                                \multicolumn{3}{l}{Is true if and only if the cell (x,y) is in the path,}\\
                                \end{tabular}\\
                        &   &\begin{tabular}{l c l}
                                \multicolumn{3}{l}{at depth d.}\\
                                    d      &$\in$  &$\{0,..,64\}$\\
                                    x,y    &$\in$  &$\{1,..,9\}$\\
                                    Digits      &:      &$1xydd$\\
                                    Range       &:      &$11100$ to $19964$\\
                                 \end{tabular}\\
    \\
    \hline
    \end{tabular*}
    \caption{Notations and variables, Secret Direction rules.}
    \label{notation:SecretDirection}
\end{table}


\begin{table}[ht!]
    \begin{tabular*}{\textwidth}{ l l @{\extracolsep{\fill}} c}
    \hline
    \\
    $\displaystyle \bigvee_{x\in\{2,3,8\}} \bigvee_{y\in\{2,3,8\}} s_{x,y,9}$ & & \consCount{SD} \label{SD-\roman{cons}}\\
    \\
    $\displaystyle \bigwedge_{x\in\{2,3,8\}} \bigwedge_{y\in\{2,3,8\}}  \bigwedge_{\substack{k\in\{3,8\}\\ k>y}} \neg s_{x,y,9} \lor \neg s_{x,k,9}$ & & \consCount{SD} \label{SD-\roman{cons}}\\
    \\
    $\displaystyle \bigwedge_{x\in\{2,3,8\}} \bigwedge_{y\in\{2,3,8\}}  \bigwedge_{\substack{k\in\{3,8\}\\ k>x}} \bigwedge_{l\in\{2,3,8\}} \neg s_{x,y,9} \lor \neg s_{k,l,9}$ & & \consCount{SD} \label{SD-\roman{cons}}\\
    \\
    $\displaystyle \bigwedge_{x\in\{2,3,8\}} \bigwedge_{y\in\{2,3,8\}}  \neg s_{x,y,9} \lor \bigvee_{d=0}^{64} s_{1,x,y,d}$ & & \consCount{SD} \label{SD-\roman{cons}}\\
    \\
    $\displaystyle \bigwedge_{x=1}^{9} \bigwedge_{y=1}^{9}  \bigwedge_{d=0}^{63}  \bigwedge_{k=d+1}^{64} \neg s_{1,x,y,d} \lor \neg s_{1,x,y,k}$ & & \consCount{SD} \label{SD-\roman{cons}}\\
    \\
    $\displaystyle \bigwedge_{d=0}^{64} \bigwedge_{x=1}^{9} \bigwedge_{y=1}^{9} \bigwedge_{k=y+1}^{9} \neg s_{1,x,y,d} \lor \neg s_{1,x,k,d}$ & & \consCount{SD} \label{SD-\roman{cons}}\\
    \\
    $\displaystyle \bigwedge_{d=0}^{64} \bigwedge_{x=1}^{9} \bigwedge_{y=1}^{9} \bigwedge_{k=x+1}^{9} \bigwedge_{l=1}^{9} \neg s_{1,x,y,d} \lor \neg s_{1,k,l,d}$ & & \consCount{SD} \label{SD-\roman{cons}}\\
    \\
    $\displaystyle \bigwedge_{d=0}^{64} \bigwedge_{x=1}^{9} \bigwedge_{y=1}^{9} \neg s_{1,x,y,d} \lor \neg s_{x,y,9}$ & s.t. $(x,y) \notin M$& \consCount{SD} \label{SD-\roman{cons}}\\
    \\
    $\displaystyle \bigwedge_{x\in\{2,3,8\}} \bigwedge_{y\in\{2,3,8\}} \bigwedge_{x'=-1}^{1} \bigwedge_{y'=-1}^{1} \bigwedge_{d=0}^{64} \neg s_{x,y,9} \lor \neg s_{1,(x+x'),(y+y'),d}$ & s.t. $(x',y') \neq (0,0)$& \consCount{SD} \label{SD-\roman{cons}}\\
    \\
    $s_{1,x_r,y_r,0}$ & & \consCount{SD} \label{SD-\roman{cons}}\\
    \\
    $\displaystyle  \bigwedge_{x_b=0}^{2} \bigwedge_{y_b=0}^{2} \bigwedge_{x'=1}^{3} \bigwedge_{y'=1}^{3} \bigwedge_{d=0}^{63} \bigwedge_{(x_n,y_n):\Psi} \bigwedge_{z=1}^{8} \mathcal{F}_1(x_b,y_b,x',y',d,x_n,y_n,z)$ && \consCount{SD} \label{SD-\roman{cons}}\\
    \\
    \hline
\end{tabular*}
    \caption{Formulae of clauses, Secret Direction rules.}
    \label{formulae:SecretDirection}
\end{table}

\FloatBarrier
\newpage
\section{Fawlty Towers}
To encode the Fawlty Tower rules, we introduce three new variables for each tower, which are true iff a tower is faulty, has increasing values or if the cell values of a tower add up to the indicated sum, respectively. Details about these three variables can be found in Table \ref{variables:FawltyTowers}. We can encode the needed increasing values of a tower by placing arrowheads between each neighbouring pair of its cells. Arrowheads can be encoded as shown in \ref{Encoding:Arrowheads}. To encode that the cells of a tower must have values that add up to the indicated sum, we can place a killer cage over them and set the cage's sum to the indicated one. The killer cage can then be encoded, as explained in \ref{encoding:killer}. Further, we can use PBCs to ensure that the correct number of towers is faulty, has increasing values or has cell values that add up to the indicated sum. The three PBCs used for this are detailed in Table \ref{notation:FawltyTowers}. The constraints derived from the Fawlty Tower rules are shown in Table \ref{Constraints:FawltyTowers}, and the formulae to encode these constraints into clauses are displayed in Table \ref{formulae:FawltyTower}.\\
\begin{table}[hb!]
    \centering
    \begin{tabular*}{\textwidth}{l @{\extracolsep{\fill}} c}
        \hline
        \\
        Constraint &Formula\\
        \\
        \hline
        \\
        \makecell[cl]{A tower in column $x$ is faulty if and only if either the\\
        sum of its cell values is not equal to the indicated one\\
        for column $x$ or its cell values are not increasing (but\\
        not both at once).} & (FT-\ref{FT-i}) and (FT-\ref{FT-ii})\\
        \\
        \makecell[cl]{If a tower has increasing cell values, all the clauses of\\
        the corresponding arrowheads placed between its cells\\
        must be true.} & (FT-\ref{FT-iii})\\
        \\
        \makecell[cl]{If the cells of a tower in column $x$ have values that add\\
        up to the sum indicated for column $x$, all the\\
        clauses of the corresponding killer cage must be true.} & (FT-\ref{FT-iv})\\
        \\
        \makecell[cl]{The number of faulty towers must be equal to the one\\
        given in the puzzle description.} & (FT-\ref{FT-v})\\
        \\
        \makecell[cl]{The number of towers with increasing cell values must\\
        be equal to the one given in the puzzle description.} & (FT-\ref{FT-vi})\\
        \\
        \makecell[cl]{The number of towers with cell values that add up to\\
        the sum indicated for the corresponding column must\\
        be equal to the one given in the puzzle description.} & (FT-\ref{FT-vii})\\
        \\
        \hline
    \end{tabular*}
        \caption{Constraints of Fawlty Tower rules.}
    \label{Constraints:FawltyTowers}
\end{table}

\begin{table}
    \centering
    \begin{tabular*}{\textwidth}{l c l}
    \hline
    \\
    \multicolumn{3}{l}{Notation:}\\
    \\
    \hline
    \\
    $Arrowhead(x,y,x',y')$      &= &\begin{tabular}{l c l}
                                            \multicolumn{3}{l}{Set of clauses needed to encode that the cell $(x,y)$ }\\
                                            \end{tabular}\\
                                        &  &\begin{tabular}{l c l}
                                    \multicolumn{3}{l}{contains a smaller value than the cell $(x',y')$ as if}\\
                                    \multicolumn{3}{l}{there were an arrowhead placed between $(x,y)$ and}\\
                                    \multicolumn{3}{l}{$(x',y')$. Arrowheads can be encoded as explained in}\\
                                    \multicolumn{3}{l}{\ref{Encoding:Arrowheads}.}\\
                            \end{tabular}\\
                                \\
    $Killer(x,y,x',y')$      &= &\begin{tabular}{l c l}
                                            \multicolumn{3}{l}{Set of clauses needed to encode that the cells between}\\
                                            \end{tabular}\\
                                        &  &\begin{tabular}{l c l}
                                    \multicolumn{3}{l}{$(x,y)$ and $(x',y')$ (including $(x,y)$ and $(x',y')$) have}\\
                                    \multicolumn{3}{l}{values that add up to the indicated sum for the tower}\\
                                    \multicolumn{3}{l}{in column $x$. This encoding is similar to the one for}\\
                                    \multicolumn{3}{l}{killer cages which is explained in \ref{encoding:killer}.}\\
                            \end{tabular}\\
        \\
    $PBC_{faulty}$            &= &\begin{tabular}{l c l}
                                    \multicolumn{3}{l}{Set of clauses needed to encode a PBC, which ensures}\\
                                    \end{tabular}\\
                                &  &\begin{tabular}{l c l}
                                    \multicolumn{3}{l}{that the number of columns containing a faulty tower}\\
                                    \multicolumn{3}{l}{matches the one stated in the puzzle description.}\\
                                    $LHS$       &=  &$\displaystyle\sum_{x=1}^9 1*faulty(x)$\\
                                    $RHS$       &=  &Number of faulty towers demanded in\\
                                                &   &puzzle description.\\
                            \end{tabular}\\
    \\
    $PBC_{inc}$            &= &\begin{tabular}{l c l}
                                    \multicolumn{3}{l}{Set of clauses needed to encode a PBC, which ensures}\\
                                    \end{tabular}\\
                                &  &\begin{tabular}{l c l}
                                    \multicolumn{3}{l}{that the number of columns containing a tower with}\\
                                    \multicolumn{3}{l}{increasing cell values matches the one stated in the}\\
                                    \multicolumn{3}{l}{puzzle description.}\\
                                    $LHS$       &=  &$\displaystyle\sum_{x=1}^9 1*inc(x)$\\
                                    $RHS$       &=  &Number of increasing towers demanded in\\
                                                &   &puzzle description.\\
                            \end{tabular}\\
    \\
    $PBC_{sum}$            &= &\begin{tabular}{l c l}
                                    \multicolumn{3}{l}{Set of clauses needed to encode a PBC, which ensures}\\
                                    \end{tabular}\\
                                &  &\begin{tabular}{l c l}
                                    \multicolumn{3}{l}{that the number of columns containing a tower with}\\
                                    \multicolumn{3}{l}{cell values that add up to the correspondingly}\\
                                    \multicolumn{3}{l}{indicated sum  matches the one stated in the puzzle}\\
                                    \multicolumn{3}{l}{description.}\\
                                    $LHS$       &=  &$\displaystyle\sum_{x=1}^9 1*sum(x)$\\
                                    $RHS$       &=  &Number of towers with a cell value sum \\
                                                &   &matching the indicated one.\\
                            \end{tabular}\\
                                \\
    $h(x)$                  &= &\begin{tabular}{l c l}
                                    \multicolumn{3}{l}{Height (number of contained cells) of tower in column}\\
                                    \end{tabular}\\
                            &  &\begin{tabular}{l c l}
                                    \multicolumn{3}{l}{$x$.}\\
                                    \end{tabular}\\  
    \\
        \hline
    \end{tabular*}
    \caption{Notation, Fawlty Tower rules.}
    \label{notation:FawltyTowers}
\end{table}

\begin{table}
    \centering
    \begin{tabular*}{\textwidth}{l c l}
    \hline
    \\
    \multicolumn{3}{l}{Variables:}\\
    \\
    \hline
    \\
    $faulty(x)$               &= &\begin{tabular}{l c l}
                                    \multicolumn{3}{l}{$s_{1,1,x,0}$}\\
                                    \end{tabular}\\
                                &  &\begin{tabular}{l c l}
                                    \multicolumn{3}{l}{True iff the tower in column $x$ is a faulty tower.}\\
                                    $x$       &$\in$  &$\{1,..,9\}$\\
                                    Digits      &:      &$11x0$\\
                                    Range       &:      &$1110$ to $1190$ 
                            \end{tabular}\\
    \\
    $inc(x)$            &= &\begin{tabular}{l c l}
                                    \multicolumn{3}{l}{$s_{1,1,x,1}$}\\
                                    \end{tabular}\\
                                &  &\begin{tabular}{l c l}
                                    \multicolumn{3}{l}{True iff the cells of the tower in column $x$ is contain increasing values}\\
                                    \multicolumn{3}{l}{from the bottom to the top.}\\
                                    $x$       &$\in$  &$\{1,..,9\}$\\
                                    Digits      &:      &$11x1$\\
                                    Range       &:      &$1111$ to $1191$ 
                            \end{tabular}\\
    \\
    $sum(x)$     &= &\begin{tabular}{l c l}
                                    \multicolumn{3}{l}{$s_{1,1,x,2}$}\\
                                    \end{tabular}\\
                                &  &\begin{tabular}{l c l}
                                    \multicolumn{3}{l}{True iff the cell values of the tower in column $x$ add up to the}\\
                                    \multicolumn{3}{l}{indicated value.}\\
                                    $x$       &$\in$  &$\{1,..,9\}$\\
                                    Digits      &:      &$11x2$\\
                                    Range       &:      &$1112$ to $1192$ 
                            \end{tabular}\\
    \\
    \hline
    \end{tabular*}
    \caption{Variables, Fawlty Tower rules.}
    \label{variables:FawltyTowers}
\end{table}

\begin{table}
    \begin{tabular*}{\textwidth}{ l l @{\extracolsep{\fill}} c}
    \hline
    \\
    \multicolumn{2}{l}{$\displaystyle \bigwedge_{x = 1}^{9}  (\neg faulty(x) \lor sum(x) \lor inc(x)) \land (\neg faulty(x) \lor \neg sum(x) \lor \neg inc(x))$} & \consCount{FT} \label{FT-\roman{cons}}\\
    \\
    \multicolumn{2}{l}{$\displaystyle \bigwedge_{x = 1}^{9}  (faulty(x) \lor sum(x) \lor \neg inc(x)) \land (faulty(x) \lor \neg sum(x) \lor inc(x))$} & \consCount{FT} \label{FT-\roman{cons}}\\
    \\
    \multicolumn{2}{l}{$\displaystyle \bigwedge_{x = 1}^{9}  \bigwedge_{y = 11-h(x)}^{9} \bigwedge_{\varphi\in Arrowhead(x,y,x,y-1)}\neg inc(x) \lor \varphi $} & \consCount{FT} \label{FT-\roman{cons}}\\
    \\
    \multicolumn{2}{l}{$\displaystyle \bigwedge_{x = 1}^{9}  \bigwedge_{y = 11-h(x)}^{9} \bigwedge_{\varphi\in Killer(x,10-h(x),x,9)}\neg sum(x) \lor \varphi $} & \consCount{FT} \label{FT-\roman{cons}}\\
    \\
    \multicolumn{2}{l}{$\displaystyle PBC_{faulty}$} & \consCount{FT} \label{FT-\roman{cons}}\\
    \\
    \multicolumn{2}{l}{$\displaystyle PBC_{inc}$} & \consCount{FT} \label{FT-\roman{cons}}\\
    \\
    \multicolumn{2}{l}{$\displaystyle PBC_{sum}$} & \consCount{FT} \label{FT-\roman{cons}}\\
    \\
    \hline
\end{tabular*}
    \caption{Formulae of clauses, Fawlty Tower rules.}
    \label{formulae:FawltyTower}
\end{table}

\FloatBarrier
\newpage
\section{Nurikabe Sudoku}
As one part of the solution to a Nurikabe Sudoku is marking island and ocean cells, we introduce corresponding variables that encode if a cell $(x,y)$ is part of the ocean or part of island $n$ (further detailed in  Table \ref{variables:NurikabeSudoku} as $\ocean(x,y)$ and  $\island(n,x,y)$). We note that there is only one ocean but multiple islands. Since an island must consist of at least three orthogonally adjacent cells and there are only 81 cells in the grid, a safe upper bound for the number of islands can be calculated by $81/3=27$. However, this number can be lowered because of the additional rule that islands may not touch each other orthogonally and that there is only one ocean. Considering this, the maximum number of islands found (by hand) to fit in the grid is 13, but we will use an upper bound of 14 for the encoding just to be sure. An example of how 13 islands can be placed in the grid that complies with the rules of Nurikabe Sudoku can be seen in Figure \ref{fig:13Islands}.\\
\\
The rule that an island must consist of at least three orthogonally adjacent cells can be broken down to each individual cell of an island by stating that each cell that is part of an island must be in a \emph{constellation} (a set) with two other neighbouring cells that are part of the same island. For such a constellation, it must hold that one of the three cells is orthogonally adjacent to the other two cells. As shown in Figure \ref{fig:Constellations}, there are 18 different possible constellations that one cell could be part of. To encode this rule, we introduce a new variable which is further explained in Table \ref{variables:NurikabeSudoku} in the definition of $\nb(c,n,x,y)$.\\

\begin{figure}
\centering
\begin{minipage}{.5\textwidth}
  \centering
  \includegraphics[width=.8\textwidth]{Figures/13Islands.png}
  \captionof{figure}{E.g. with 13 islands}
  \label{fig:13Islands}
\end{minipage}%
\begin{minipage}{.5\textwidth}
  \centering
  \includegraphics[width=.8\textwidth]{Figures/DeepestFlood.png}
  \captionof{figure}{E.g. with max. flood depth 49 }
  \label{fig:DeepestFlood}
\end{minipage}
\end{figure}

Further, we must enforce that islands are continuous. So from every cell that is part of an island  $n$ it must be possible to reach any other cell of island $n$, without crossing an ocean cell. This reachability can be enforced by constraints describing a Floodfill methodology. However, we will call this \emph{walk} as the word \emph{flood} is better suited to describe the similar idea used for the ocean cells. A walk is defined by its corresponding island $n$ and its source $(x_s,y_s)$, which is said to be at depth 1 of the walk. Outgoing from the source, orthogonally adjacent cells are walked with increasing depth numbers. As we model a Floodfill multiple cells can be at the same depth of a walk. At depth 9, the walk will have covered all cells of an island because the rules state that there are no value repetitions within an island which limits the size of an island to at most nine cells. The variable used to encode walks is further described in Table \ref{variables:NurikabeSudoku} in the definition of $\walk(d,n,x_s,y_s,x,y)$.\\

The rules also dictate that there may only be one continuous ocean, which can be enforced with an approach similar to the one used for the continuousness of islands. However, as value repetitions are allowed within the ocean, its size is not limited to nine cells.The depth needed by a flood to cover all ocean cells is maximized if there are as few cells as possible in every depth. With only one cell in each depth, an upper bound of 81 can be given as the grid contains a total of 81 cells. But this number can be lowered as more than one cell will be at the same depth in practice. After experimenting with possible island placements by hand, we found an upper limit for the flood depth of 49. An Example for an island placements that requires this flood depth can be seen in Figure \ref{fig:DeepestFlood}. The variable used to encode floods is further described in Table \ref{variables:NurikabeSudoku} in the definition of $\flood(d,x_s,y_s,x,y)$.\\
\\
The constraints to encode Nurikabe Sudoku are stated in the Figures \ref{Constraints:NurikabeSudoku1}, \ref{Constraints:NurikabeSudoku2} and \ref{Constraints:NurikabeSudoku3}. How the constraints are encoded into clauses is shown in the Figures \ref{formulae:NurikabeSudoku1}, \ref{formulae:NurikabeSudoku2} and \ref{formulae:NurikabeSudoku3}.\\


\begin{table}[hb!]
    \centering
    \begin{tabular*}{\textwidth}{l @{\extracolsep{\fill}} c}
        \hline
        \\
        Constraint &Formula\\
        \\
        \hline
        \\
        \makecell[cl]{No $(2 \times 2)$-square can only consist of ocean cells.} & (NK-\ref{NK-i})\\
        \\
        \makecell[cl]{Cells that are part of the same island can not have the\\
        same value.} & (NK-\ref{NK-ii}) and (NK-\ref{NK-iii})\\
        \\
        \makecell[cl]{Island cells of different islands can not touch each\\
        other orthogonally.} & (NK-\ref{NK-iv}) and (NK-\ref{NK-v})\\
        \\
        \makecell[cl]{Island cells must be in at least one constellation with\\
        two neighbouring cells.} & (NK-\ref{NK-vi})\\
        \\
        \makecell[cl]{If cell $(x,y)$ is part of island $n$ and is in a constellation\\
        that includes the two neighbouring cells $(x',y')$ and\\
        $(x'',y'')$ then the two neighbouring cells must be part\\
        of the same island $n$.} & (NK-\ref{NK-vii}) and (NK-\ref{NK-viii})\\
        \\
        \makecell[cl]{Cells must be part of the ocean or part of at least one\\
        island.} & (NK-\ref{NK-ix})\\
        \\
        \makecell[cl]{Cells that are part of the ocean, are not part of any\\
        island.} & (NK-\ref{NK-x})\\
        \\
        \makecell[cl]{Cells can be part of at most one island.} & (NK-\ref{NK-xi})\\
        \\
        \hline
    \end{tabular*}
        \caption{Constraints-1 of Nurikabe Sudoku rules.}
    \label{Constraints:NurikabeSudoku1}
\end{table}

\begin{table}
    \centering
    \begin{tabular*}{\textwidth}{l @{\extracolsep{\fill}} c}
        \hline
        \\
        \makecell[cl]{An ocean cell $(x,y)$ must be in at least one depth of\\
        any flood that has a source $(x_s,y_s)$ which is also an\\
        ocean cell.} & (NK-\ref{NK-xii})\\
        \\
        \makecell[cl]{Cells can be in at most one depth per flood with\\
        source $(x_s,y_s)$.} & (NK-\ref{NK-xiii})\\
        \\
        \makecell[cl]{Cells that are not part of the ocean are not flooded.} & (NK-\ref{NK-xiv})\\
        \\
        \makecell[cl]{If cell (x,y) is part of the ocean, it is in depth 1 of the\\
        flood with source (x,y).} & (NK-\ref{NK-xv})\\
        \\
        \makecell[cl]{Only the source cell of a flood can be in depth 1 of a\\
        flood (where it is source).} & (NK-\ref{NK-xvi})\\
        \\
        \makecell[cl]{If a cell $(x,y)$ is in depth d of a flood with source\\
        $(x_s,y_s)$, and $(x,y) \neq (x_s,y_s)$, and the cell $(x_s,y_s)$ is\\
        part of the ocean, then there must be an orthogonally\\
        adjacent cell to $(x,y)$ which is in depth $d-1$ of the\\
        flood with source $(x_s,y_s)$.} & (NK-\ref{NK-xvii})\\
        \\
        \makecell[cl]{If a cell $(x,y)$ is part of island $n$, then the cell (x,y) is\\
        at depth 1 of the walk on island $n$ with source (x,y).} & (NK-\ref{NK-xviii})\\
        \\
        \makecell[cl]{Only the source cell of a walk can be in depth 1 of a\\
        walk (where it is source).} & (NK-\ref{NK-xix})\\
        \\
        \makecell[cl]{If a cell $(x,y)$ is in depth d of a walk on island $n$ with\\
        source $(x_s,y_s)$, and $(x,y) \neq (x_s,y_s)$, and the cell\\
        $(x_s,y_s)$ is part of island $n$, then there must be an\\
        orthogonally adjacent cell to $(x,y)$ which is in depth\\
        $d-1$ of the walk on island $n$ with source $(x_s,y_s)$.} & (NK-\ref{NK-xx})\\
        \\
        \makecell[cl]{If cell $(x,y)$ is part of island $n$ and cell $(x_s,y_s)$ is part\\
        of island $n$, then cell $(x,y)$ must be in at least one\\
        depth of the walk on island $n$ that has source $(x_s,y_s)$.} & (NK-\ref{NK-xxi})\\
        \\
        \makecell[cl]{A cell $(x,y)$ can be in at most one depth of a walk on\\
        an island $n$ that has source $(x_s,y_s)$.} & (NK-\ref{NK-xxii})\\
        \\
        \makecell[cl]{Cells that are not part of an island $n$ can not be in\\
        any depth of a walk on island $n$.} & (NK-\ref{NK-xxiii})\\
        \\
        \hline
    \end{tabular*}
        \caption{Constraints-2 of Nurikabe Sudoku rules.}
    \label{Constraints:NurikabeSudoku2}
\end{table}

\begin{table}
    \centering
    \begin{tabular*}{\textwidth}{l @{\extracolsep{\fill}} c}
        \hline
        \\
        \makecell[cl]{If a cell that is part of the ocean contains an arrow,\\
        and has value $z$, then there are $z$ cells that are part\\
        of the ocean in the direction the arrow points.} & (NK-\ref{NK-xxiv})\\
        \\
        \makecell[cl]{If a cell that is not part of the ocean contains an\\
        arrow, and has value $z$, then there are $z$ cells that are\\
        not part of the ocean in the direction the arrow points.} & (NK-\ref{NK-xxv})\\
        \\
        \makecell[cl]{There is no solution, if a cell that is part of the grid's\\
        edge contains an arrow that points directly outside\\
        the grid.} & (NK-\ref{NK-xxvi})\\
        \\
        \hline
    \end{tabular*}
        \caption{Constraints-3 of Nurikabe Sudoku rules.}
    \label{Constraints:NurikabeSudoku3}
\end{table}

\begin{figure}
\centering

  \centering
  \includegraphics[width=.5\textwidth]{Figures/ConstelationsBW.png}
  \captionsetup{justification=centering,margin=0cm}
  \captionof{figure}{The 18 possible constellations, depicted for the center cell}
  \label{fig:Constellations}

\end{figure}

\begin{table}
    \centering
    \begin{tabular*}{\textwidth}{l c l}
    \hline
    \\
    \multicolumn{3}{l}{Variables:}\\
    \\
    \hline
    \\
    $\ocean(x,y)$               &= &\begin{tabular}{l c l}
                                    \multicolumn{3}{l}{$s_{1,5,0,0,0,x,y}$}\\
                                    \end{tabular}\\
                                &  &\begin{tabular}{l c l}
                                    \multicolumn{3}{l}{True iff cell $(x,y)$ is part of the ocean.}\\
                                    $x,y$       &$\in$  &$\{1,..,9\}$\\
                                    Digits      &:      &$15000xy$\\
                                    Range       &:      &$1500011$ to $1500099$ 
                            \end{tabular}\\
    \\
    $\island(n,x,y)$            &= &\begin{tabular}{l c l}
                                    \multicolumn{3}{l}{$s_{1,4,0,n,x,y}$}\\
                                    \end{tabular}\\
                                &  &\begin{tabular}{l c l}
                                    \multicolumn{3}{l}{True iff cell $(x,y)$ is part of island $n$.}\\
                                    $x,y$   &$\in$  &$\{1,..,9\}$\\
                                    $n$     &$\in$  &$\{1,..,14\}$\\
                                    Digits  &:      &$140nnxy$\\
                                    Range   &:      &$1400111$ to $1401499$\\
                            \end{tabular}\\
    \\
    $\flood(d,x_s,y_s,x,y)$     &= &\begin{tabular}{l c l}
                                    \multicolumn{3}{l}{$s_{2,d,x_s,y_s,x,y}$}\\
                                    \end{tabular}\\
                                &  &\begin{tabular}{l c l}
                                    \multicolumn{3}{l}{True iff cell $(x,y)$ is in flood-depth $d$}\\
                                    \multicolumn{3}{l}{of the flood with source $(x_s,y_s)$.}\\
                                    $x,y$       &$\in$  &$\{1,..,9\}$\\
                                    $x_s,y_s$   &$\in$  &$\{1,..,9\}$\\
                                    $d$         &$\in$  &$\{1,..,49\}$\\
                                    Digits      &:      &$2ddx_sy_sxy$\\
                                    Range       &:      &$2011111$ to $2499999$\\
                            \end{tabular}\\
    \\
    $\walk(d,n,x_s,y_s,x,y)$    &= &\begin{tabular}{l c l}
                                    \multicolumn{3}{l}{$s_{1,d,n,x_s,y_s,x,y}$}\\
                                    \end{tabular}\\
                                &  &\begin{tabular}{l c l}
                                    \multicolumn{3}{l}{True iff cell $(x,y)$ is in depth $d$ of the}\\
                                    \multicolumn{3}{l}{walk on island $n$ with source $(x_s,y_s)$.}\\
                                    $x,y$       &$\in$  &$\{1,..,9\}$\\
                                    $x_s,y_s$   &$\in$  &$\{1,..,9\}$\\
                                    $d$         &$\in$  &$\{1,..,9\}$\\
                                    $n$         &$\in$  &$\{1,..,14\}$\\
                                    Digits      &:      &$1dnnx_sy_sxy$\\
                                    Range       &:      &$11011111$ to $19149999$\\
                            \end{tabular}\\
    \\
    $\nb(c,n,x,y)$              &= &\begin{tabular}{l c l}
                                    \multicolumn{3}{l}{$s_{1,0,c,n,x,y}$}\\
                                    \end{tabular}\\
                                &  &\begin{tabular}{l c l}
                                    \multicolumn{3}{l}{True iff cell $(x,y)$ belongs to island $n$,}\\
                                    \multicolumn{3}{l}{and has two adjacent neighbours that}\\
                                    \multicolumn{3}{l}{also belong to  island $n$, and that are}\\
                                    \multicolumn{3}{l}{in constellation $c$ with cell $(x,y)$.}\\
                                    $c$         &$\in$  &$\{1,..,18\}$\\
                                    $n$         &$\in$  &$\{1,..,14\}$\\
                                    $x,y$       &$\in$  &$\{1,..,9\}$\\
                                    Digits      &:      &$1ccnnxy$\\
                                    Range       &:      &$1010111$ to $1181499$\\
                            \end{tabular}\\
    \\
    \hline
    \end{tabular*}
    \caption{Variables, Nurikabe Sudoku rules.}
    \label{variables:NurikabeSudoku}
\end{table}

\begin{table}
    \centering
    \begin{tabular*}{\textwidth}{l c l}
    \hline
    \\
    \multicolumn{3}{l}{Notation:}\\
    \\
    \hline
    \\
    $\mathcal{A}$           &= & $\{(x,y)~|~(x,y)~\textnormal{is grid cell that contains an arrow}\}$\\
    \\
    $\mathcal{B}(x,y)$      &= & $\{(x+1,y), (x-1,y), (x,y+1), (x,y-1)\}$\\
    \\
    $\mathcal{G}$           &= & $\{(a,b)~|~a,b\in \{1,2,3,4,5,6,7,8,9\}\}$\\
    \\
    $\mathcal{F}_2(d,x,y,x_s,y_s)$          &= & $\displaystyle \neg \ocean(x_s,y_s) \lor \bigvee_{\mathclap{(x',y'):\mathcal{B}(x,y)}} \flood(d-1,x_s,y_s,x',y')~~s.t.~~(x',y') \in \mathcal{G}$\\
        \\
    $\mathcal{F}_3(d,n,x,y,x_s,y_s)$        &= & $\displaystyle \neg \island(n,x_s,y_s) \lor \bigvee_{\mathclap{(x',y'):\mathcal{B}(x,y)}} \walk(d-1,n,x_s,y_s,x',y')~~s.t.~~(x',y') \in \mathcal{G}$\\
    \\
    $\eta(c,x,y) $   &= &$\begin{cases}
    ((x,y-1),(x,y-2))       & c = 1 \\
    ((x,y-1),(x+1,y-1))     & c = 2 \\
    ((x+1,y),(x+1,y-1))     & c = 3 \\
    ((x+1,y),(x+2,y))       & c = 4 \\
    ((x+1,y),(x+1,y+1))     & c = 5 \\
    ((x,y+1),(x+1,y+1))     & c = 6 \\
    ((x,y+1),(x,y+2))       & c = 7 \\
    ((x,y+1),(x-1,y+1))     & c = 8 \\
    ((x-1,y),(x-1,y+1))     & c = 9 \\
    ((x-1,y),(x-2,y))       & c = 10 \\
    ((x-1,y),(x-1,y-1))     & c = 11 \\
    ((x,y-1),(x-1,y-1))     & c = 12 \\
    ((x,y-1),(x+1,y))       & c = 13 \\
    ((x,y-1),(x,y+1))       & c = 14 \\
    ((x,y-1),(x-1,y))       & c = 15 \\
    ((x+1,y),(x,y+1))       & c = 16 \\
    ((x-1,y),(x+1,y))       & c = 17 \\
    ((x-1,y),(x,y+1))       & c = 18 \\
    \end{cases}$\\
    \\
    \hline
    \end{tabular*}
    \caption{Notations, Nurikabe Sudoku rules.}
    \label{notation:NurikabeSudoku}
\end{table}

\begin{table}
    \centering
    \begin{tabular*}{\textwidth}{l c l}
    \hline
    \\
    $PBC_O(x,y,z)$          &= & Set of clauses needed to encode that in the direction where\\
                            &  & the arrow of cell (x,y) points, there are $z$ cells that are part\\
                            &  & of the ocean.\\
                            &  & LHS = $\begin{cases}
                                        \displaystyle\sum_{\substack{k \in \{1,...,8\}\\k<y}} 1*\ocean(x,k)       & (x,y) \textnormal{ contains}  \uparrow\\
                                        \\
                                        \displaystyle\sum_{k=y+1}^{9} 1*\ocean(x,k)       & (x,y) \textnormal{ contains}  \downarrow\\
                                        \\
                                        \displaystyle\sum_{\substack{k\in\{1,...,8\}\\k<x}} 1*\ocean(k,y)       & (x,y) \textnormal{ contains}  \leftarrow\\
                                        \\
                                        \displaystyle\sum_{k=x+1}^{9} 1*\ocean(k,y)       & (x,y) \textnormal{ contains}  \rightarrow\\
                                        \end{cases}$\\
                            &  & RHS = $z$\\
    \\
    $PBC_I(x,y,z)$          &= & Set of clauses needed to encode that in the direction where\\
                            &  & the arrow of cell (x,y) points, there are $z$ cells that are not\\
                            &  & part of the ocean.\\
                            &  & LHS = $\begin{cases}
                                        \displaystyle\sum_{\substack{k \in \{1,...,8\}\\k<y}} 1*\neg\ocean(x,k)       & (x,y) \textnormal{ contains}  \uparrow\\
                                        \\
                                        \displaystyle\sum_{k=y+1}^{9} 1*\neg\ocean(x,k)       & (x,y) \textnormal{ contains}  \downarrow\\
                                        \\
                                        \displaystyle\sum_{\substack{k\in\{1,...,8\}\\k<x}} 1*\neg\ocean(k,y)       & (x,y) \textnormal{ contains}  \leftarrow\\
                                        \\
                                        \displaystyle\sum_{k=x+1}^{9} 1*\neg\ocean(k,y)       & (x,y) \textnormal{ contains}  \rightarrow\\
                                        \end{cases}$\\
                            &  & RHS = $z$\\
                            &  & (As we only defined PBCs for positive literals, we must in\\
                            &  &practice introduce an additional variable $s_v$ for each\\
                            &  &$\neg\ocean(x,y)$. The corresponding variable  $s_v$ is defined to\\
                            &  &be true iff $\neg\ocean(x,y)$ is true (iff $\ocean(x,y)$ is false).)\\
    \\
    \hline
    \end{tabular*}
    \caption{PBCs, Nurikabe Sudoku rules.}
    \label{PBC:NurikabeSudoku}
\end{table}

\begin{table}[ht!]
    \begin{tabular*}{\textwidth}{ l l @{\extracolsep{\fill}} c}
    \hline
    \\
    \multicolumn{2}{l}{$\displaystyle \bigwedge_{x = 1}^{8} \bigwedge_{y = 1}^{8} \neg \ocean(x,y) \lor \neg \ocean(x+1,y) \lor \neg \ocean(x,y+1) \lor \neg \ocean(x+1,y+1)$} & \consCount{NK} \label{NK-\roman{cons}}\\
    \\
    \multicolumn{2}{l}{$\displaystyle \bigwedge_{n = 1}^{14} \bigwedge_{x = 1}^{9} \bigwedge_{y = 1}^{9} \bigwedge_{z = 1}^{9} \bigwedge_{k = y+1}^{9} \neg \island(n,x,y) \lor \neg \island(n,x,k) \lor \neg s_{x,y,z} \lor \neg s_{x,k,z}$} & \consCount{NK} \label{NK-\roman{cons}}\\
    \\
    \multicolumn{2}{l}{$\displaystyle \bigwedge_{n = 1}^{14} \bigwedge_{x = 1}^{9} \bigwedge_{y = 1}^{9} \bigwedge_{z = 1}^{9} \bigwedge_{k = x+1}^{9} \bigwedge_{l = 1}^{9} \neg \island(n,x,y) \lor \neg \island(n,k,l) \lor \neg s_{x,y,z} \lor \neg s_{k,l,z}$}& \consCount{NK} \label{NK-\roman{cons}}\\
    \\
    \multicolumn{2}{l}{$\displaystyle \bigwedge_{n = 1}^{14} \bigwedge_{x = 1}^{8} \bigwedge_{y = 1}^{9} \bigwedge_{\substack{k = 1\\ k\neq n}}^{14} \neg \island(n,x,y) \lor \neg \island(k,x+1,y)$} & \consCount{NK} \label{NK-\roman{cons}}\\
    \\
    \multicolumn{2}{l}{$\displaystyle \bigwedge_{n = 1}^{14} \bigwedge_{x = 1}^{9} \bigwedge_{y = 1}^{8} \bigwedge_{\substack{k = 1\\ k\neq n}}^{14} \neg \island(n,x,y) \lor \neg \island(k,x,y+1)$} & \consCount{NK} \label{NK-\roman{cons}}\\
    \\
    $\displaystyle \bigwedge_{n = 1}^{14} \bigwedge_{x = 1}^{9} \bigwedge_{y = 1}^{9} \neg \island(n,x,y) \lor \bigvee_{\substack{c=1\\ ((x',y'),(x'',y''))=\eta(c,x,y)}}^{18} \nb(c,n,x,y)$ &s.t.$\substack{(x'',y'') \in \mathcal{G}\\ (x',y') \in \mathcal{G}}$ & \consCount{NK} \label{NK-\roman{cons}}\\
    \\
    $\displaystyle \bigwedge_{n = 1}^{14} \bigwedge_{x = 1}^{9} \bigwedge_{y = 1}^{9} \bigwedge_{\substack{c=1\\ ((x',y'),(x'',y''))=\eta(c,x,y)}}^{18} \neg \nb(c,n,x,y) \lor \island(n,x',y')$ &s.t.$\substack{(x'',y'') \in \mathcal{G}\\ (x',y') \in \mathcal{G}}$ & \consCount{NK} \label{NK-\roman{cons}}\\
    \\
    $\displaystyle \bigwedge_{n = 1}^{14} \bigwedge_{x = 1}^{9} \bigwedge_{y = 1}^{9} \bigwedge_{\substack{c=1\\ ((x',y'),(x'',y''))=\eta(c,x,y)}}^{18} \neg\nb(c,n,x,y) \lor \island(n,x'',y'')$ &s.t.$\substack{(x'',y'') \in \mathcal{G}\\ (x',y') \in \mathcal{G}}$ & \consCount{NK} \label{NK-\roman{cons}}\\
    \\
    $\displaystyle \bigwedge_{x = 1}^{9} \bigwedge_{y = 1}^{9} \ocean(x,y) \lor \bigvee_{n = 1}^{14} \island(n,x,y)$ & & \consCount{NK} \label{NK-\roman{cons}}\\
    \\
    $\displaystyle \bigwedge_{n = 1}^{14} \bigwedge_{x = 1}^{9} \bigwedge_{y = 1}^{9} \neg \ocean(x,y) \lor \neg \island(n,x,y)$ & & \consCount{NK} \label{NK-\roman{cons}}\\
    \\
    $\displaystyle \bigwedge_{n = 1}^{14} \bigwedge_{x = 1}^{9} \bigwedge_{y = 1}^{9} \bigwedge_{k = n+1}^{14} \neg \island(n,x,y) \lor \neg \island(k,x,y)$ & & \consCount{NK} \label{NK-\roman{cons}}\\
    \\
    \hline
\end{tabular*}
    \caption{Formulae-1 of clauses, Nurikabe Sudoku rules.}
    \label{formulae:NurikabeSudoku1}
\end{table}

\begin{table}[ht!]
    \begin{tabular*}{\textwidth}{ l l @{\extracolsep{\fill}} c}
    \hline
    \\
    $\displaystyle \bigwedge_{x = 1}^{9} \bigwedge_{y = 1}^{9} \bigwedge_{x_s = 1}^{9} \bigwedge_{y_s = 1}^{9} \neg \ocean(n,x_s,y_s) \lor \neg \ocean(n,x,y) \lor \bigvee_{d=1}^{49} \flood(d,x_s,y_s,x,y)$ & & \consCount{NK} \label{NK-\roman{cons}}\\
    \\
    $\displaystyle \bigwedge_{x = 1}^{9} \bigwedge_{y = 1}^{9} \bigwedge_{x_s = 1}^{9} \bigwedge_{y_s = 1}^{9} \bigwedge_{d=1}^{49} \bigwedge_{k=d+1}^{49} \neg \flood(d,x_s,y_s,x,y) \lor \neg \flood(k,x_s,y_s,x,y)$ & & \consCount{NK} \label{NK-\roman{cons}}\\
    \\
    $\displaystyle \bigwedge_{x = 1}^{9} \bigwedge_{y = 1}^{9} \bigwedge_{x_s = 1}^{9} \bigwedge_{y_s = 1}^{9} \bigwedge_{d=1}^{49} \ocean(x,y) \lor \neg \flood(d,x_s,y_s,x,y)$ & & \consCount{NK} \label{NK-\roman{cons}}\\
    \\
    $\displaystyle \bigwedge_{x = 1}^{9} \bigwedge_{y = 1}^{9} \neg \ocean(x,y) \lor \flood(1,x,y,x,y)$ & & \consCount{NK} \label{NK-\roman{cons}}\\
    \\
    $\displaystyle \bigwedge_{x = 1}^{9} \bigwedge_{y = 1}^{9} \bigwedge_{x_s = 1}^{9} \bigwedge_{y_s = 1}^{9} \neg \flood(1,x_s,y_s,x,y)$~~~~with $(x,y) \neq (x_s,y_s)$ & & \consCount{NK} \label{NK-\roman{cons}}\\
    \\
    $\displaystyle \bigwedge_{x = 1}^{9} \bigwedge_{y = 1}^{9} \bigwedge_{x_s = 1}^{9} \bigwedge_{\substack{y_s = 1\\ (x,y) \neq (x_s,y_s)}}^{9} \bigwedge_{d = 2}^{49} \neg \flood(d,x_s,y_s,x,y) \lor \mathcal{F}_2(d,x,y,x_s,y_s) $ & & \consCount{NK} \label{NK-\roman{cons}}\\
    \\
    $\displaystyle \bigwedge_{n = 1}^{14} \bigwedge_{x = 1}^{9} \bigwedge_{y = 1}^{9} \neg \island(n,x,y) \lor \walk(1,n,x,y,x,y)$ & & \consCount{NK} \label{NK-\roman{cons}}\\
    \\
    $\displaystyle \bigwedge_{n = 1}^{14} \bigwedge_{x = 1}^{9} \bigwedge_{y = 1}^{9} \bigwedge_{x_s = 1}^{9} \bigwedge_{\substack{y_s = 1\\ (x,y) \neq (x_s,y_s)}}^{9} \neg \walk(1,n,x_s,y_s,x,y)$ & & \consCount{NK} \label{NK-\roman{cons}}\\
    \\
    $\displaystyle \bigwedge_{n = 1}^{14} \bigwedge_{x = 1}^{9} \bigwedge_{y = 1}^{9} \bigwedge_{x_s = 1}^{9} \bigwedge_{\substack{y_s = 1\\ (x,y) \neq (x_s,y_s)}}^{9} \bigwedge_{d = 2}^{9} \neg \walk(d,n,x_s,y_s,x,y) \lor \mathcal{F}_3(d,n,x,y,x_s,y_s) $ & & \consCount{NK} \label{NK-\roman{cons}}\\
    \\
    $\displaystyle \bigwedge_{n = 1}^{14} \bigwedge_{x = 1}^{9} \bigwedge_{y = 1}^{9} \bigwedge_{x_s = 1}^{9} \bigwedge_{y_s = 1}^{9} \neg \island(n,x_s,y_s) \lor \neg \island(n,x,y) \lor \bigvee_{d = 1}^{9} \walk(d,n,x_s,y_s,x,y) $ & & \consCount{NK} \label{NK-\roman{cons}}\\
    \\
    $\displaystyle \bigwedge_{n = 1}^{14} \bigwedge_{x = 1}^{9} \bigwedge_{y = 1}^{9} \bigwedge_{x_s = 1}^{9} \bigwedge_{y_s = 1}^{9} \bigwedge_{d = 1}^{9} \bigwedge_{k = d+1}^{9} \neg  \walk(d,n,x_s,y_s,x,y) \lor \neg \walk(k,n,x_s,y_s,x,y) $ & & \consCount{NK} \label{NK-\roman{cons}}\\
    \\
    $\displaystyle \bigwedge_{n = 1}^{14} \bigwedge_{x = 1}^{9} \bigwedge_{y = 1}^{9} \bigwedge_{x_s = 1}^{9} \bigwedge_{y_s = 1}^{9} \bigwedge_{d = 1}^{9} \island(n,x,y) \lor \neg \walk(d,n,x_s,y_s,x,y)$ & & \consCount{NK} \label{NK-\roman{cons}}\\
    \\
    \hline
\end{tabular*}
    \caption{Formulae-2 of clauses, Nurikabe Sudoku rules.}
    \label{formulae:NurikabeSudoku2}
\end{table}

\begin{table}[ht!]
    \begin{tabular*}{\textwidth}{ l l @{\extracolsep{\fill}} c}
    \hline
     \\
    $\displaystyle \bigwedge_{(x,y):\mathcal{A}} \bigwedge_{z = 1}^{9} \bigwedge_{\varphi \in PBC_O(x,y,z)} \neg \ocean(x,y) \lor \neg s_{x,y,z} \lor \varphi $ & & \consCount{NK} \label{NK-\roman{cons}}\\
    \\
    $\displaystyle \bigwedge_{(x,y):\mathcal{A}} \bigwedge_{z = 1}^{9} \bigwedge_{\varphi \in PBC_I(x,y,z)} \ocean(x,y) \lor \neg s_{x,y,z} \lor \varphi$ & & \consCount{NK} \label{NK-\roman{cons}}\\
    \\
    $\square$ if the grid contains an arrow that points directly outside the grid. & & \consCount{NK} \label{NK-\roman{cons}}\\
    \\
    \hline
\end{tabular*}
    \caption{Formulae-3 of clauses, Nurikabe Sudoku rules.}
    \label{formulae:NurikabeSudoku3}
\end{table}