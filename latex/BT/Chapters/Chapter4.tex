% !TEX root = ../Thesis.tex
\chapter{Encoding}\label{Encoding}
As introduced in \ref{DIMACS}, the SAT-Solvers expect a DIMACS file as input that describes a set of clauses. This chapter details how the different Sudoku Variants and their rules can be encoded into these clause sets. We elaborate on the different variants separately, but as long as there are no direct contradictions, the different variants and rules could be freely combined (which can be done by creating the union of the corresponding clause sets). In the following formulae, we use the notation $s_{x_{n},...,x_{0}}$ to describe boolean variables, the corresponding integer numbers in the DIMACS files have the values $x_{n}*10^{n}+...+x_{0}*10^{0}$. For example, the literal $\neg s_{1,2,3}$ would be transformed to $-123$. There are two ways how we choose the name (number) for a variable during the encoding process:

\begin{enumerate}
    \item We use increasing values starting from 1. These dynamically chosen values are, by example, used for variables necessary to encode PBCs. We use $s_{v}$, for $v \in \mathbb{N}$, to describe them in the formulae as we do not know the corresponding integers in advance.
    \item  We use fixed intervals of numbers to encode certain constraints. Here every digit of a number has a semantic meaning. The dynamically generated variables skip these fixed intervals. For example, a fixed interval is used for the boolean variables that describe the cell values of the Sudoku grid. The variable $s_{x,y,z}$ is true if and only if cell ($x$,$y$) has the value $z$ assigned (as proposed in \cite{Lynce2006SudokuAsASATProblem}). The other used fixed intervals are indicated in the corresponding subsections of this chapter\todo{Actually do that!}. 
\end{enumerate}

It is important to note that the numbers used as variable names are not ``continuous". There are ``gaps". For example, not all integers from 1 to 1000 are used. This is because the dynamic variables skip the entire interval from 111 to 999, and the fixed variables used to encode cell values will not use numbers like 400 because we start to count rows at 1, and the grid cells can only hold values from 1 to 9. The needed fixed intervals grow larger when encoding more complicated variants, and with them often also the ``gaps". These gaps have no effect logically, but they can have a non-negligible effect on the time needed by the solvers, which is affected by the highest integer value used to describe a variable. However, as the ``gaps" are the same when encoding Sudokus with the same rules, they should not make a difference when comparing solver-times.

\section{Encoding of PBCs using Binary Decision Diagrams}\label{PBCEncodingBDD}
A binary decision diagram (BDD) is a directed graph that can be used to represent logical formulae and PBCs. 
Variables are considered in a fixed order. As discussed in \cite{Een2006TranslatingPC}, ordering them by decreasing\todo{the otherway arround is not significatly slower, sometimes even faster!} weight values is generally reasonable. Every graph node corresponds to a sum and a subset of variables of the formula. The BDD has a root node corresponding to the empty variable set and sum 0. If a node is $n$ edges away from the root, its variable set contains the $n$ first variables, and such a node is said to be at depth $n$. An edge from a node $u$ at depth $k$ to a node $v$ at depth $k+1$ corresponds to assigning a truth value to the $(k+1)$-th variable. Every node has at most two successors, one reachable via the edge that corresponds to assigning \true{} and one via the edge that corresponds to assigning \false{} to the next variable. Nodes store references to their successors as \emph{$\textit{true}_{child}$} and \emph{$\textit{false}_{child}$}, respectively. Paths from the root to a node correspond to variable assignments to the variables in a node´s variable set and define the node's sum, which is equal to the sum of PBC weights that are multiplied by variables that are set to \true{} by the assignment.\\

Terminal nodes however have no children. Terminal nodes are either nodes at depth $l$ (with $l$ equal to the total number of different variables in the formula) or nodes with a too high or too low sum regarding the variables already assigned in the paths to them.\\

An integer number is assigned to each node which can later be used to describe a boolean variable (called \emph{extendable}). The  \emph{extendable} variable of a node is true if and only if the partial assignments that correspond to the paths from the root to it can be further extended to total assignments that respect the PBC the BDD represents.\\

The BDD can be built using a Breadth-First-Search starting in the root, shown as pseudocode in \ref{CodeBDDConstruction}. The version we use for encoding PBCs differs from the one introduced in \cite{Een2006TranslatingPC} because it is written to encode equations rather than inequations. Further, the used queue has additional functionalities: Given a node, it can check if it already contains a node with the same attribute values and can return said equal node. The \emph{updated sums} of successor nodes either are the same as their predecessors (\false{} was assigned) or are equal to the sum of their predecessor plus the weight value corresponding to the edge that led to them (\true{} was assigned).



{
\pseudo{}
\renewcommand{\lstlistingname}{Algorithm}
\begin{lstlisting}[frame=single,caption={Pseudo Code of BDD construction},captionpos=b, label=CodeBDDConstruction, basicstyle=\footnotesize]
BuildBDD(PBC):
    queue = empty queue
    queue.append(createRoot())
    While not queue is empty:
        Node current = queue.first()
        If assigning next variable leads to a total assignment:
            Node cT = Create true successor node with updated sum
            Node cF = Create true successor node with updated sum
        Else:
            Node cT = Create true successor node with updated sum
            If cT.sum >= RHS:
                # true successor is a terminal node
            Else:
                # true successor is not a terminal node
                If cT not in queue:
                    queue.append(cT)
                Else:
                    cT = queue.get(cT)
            Node cF = Create false successor node with updated sum
            If cF.sum + sum of remaining weights < RHS:
                # false successor is a terminal node
            Else:
                # false successor is not a terminal node
                If cF not in queue:
                    queue.append(cF)
                Else:
                    cF = queue.get(cF)
        current.true_child = cT
        current.false_child = cF
    Return root
\end{lstlisting}
}

Once the BDD is built, we can transform it into clauses. In \cite{Een2006TranslatingPC}, it is explained how this can be achieved by treating the BDD network as a circuit of ITEs (if-then-else gates). However, it suffices to know that the BDD can be transformed by doing a second Breadth-First-Search starting from the root and that for each visited node that is not a terminal node, the following six implications must be added to the set of formulas \cite{Een2006TranslatingPC}:

\begin{enumerate}
    \item If the \emph{extendable} variable of the current node is true and the variable corresponding to the leaving edges from this node is true, then it follows that the \emph{extendable} variable of the $true_{child}$ node is true.
    \item If the \emph{extendable} variable of the current node is true and the variable corresponding to the leaving edges from this node is false, then it follows that the \emph{extendable} variable of the $false_{child}$ node is true.
    \item If the \emph{extendable} variable of the current node is false and the variable corresponding to the leaving edges from this node is true, then it follows that the \emph{extendable} variable of the $true_{child}$ node is false.
    \item If the \emph{extendable} variable of the current node is false and the variable corresponding to the leaving edges from this node is false, then it follows that the \emph{extendable} variable of the $false_{child}$ node is false.
    \item If the \emph{extendable} variable of the $true_{child}$ node is true and the \emph{extendable} variable of the $false_{child}$ node is true, then it follows that the \emph{extendable} variable of the current node is true.
    \item If the \emph{extendable} variable of the $true_{child}$ node is false and the \emph{extendable} variable of the $false_{child}$ node is false, then it follows that \emph{extendable} variable of the current node is false.
\end{enumerate}

Additionally, we add a clause that only contains the positive literal corresponding to the \emph{extendable} variable of the root node. Also, when visiting a node during this second Breadth-First-Search, we only append its children that are not terminal nodes. For children that are terminal nodes, we add a clause to the set of clauses: 
\begin{itemize}
    \item If the child's sum is equal to the RHS, a clause containing a positive literal corresponding to the \emph{extendable} variable of the child is added.
    \item If the child's sum is unequal the RHS, a clause containing a negative literal corresponding to the \emph{extendable} variable of the child is added.
\end{itemize}
An example is shown in Figure \ref{fig:BDDExample} where the BDD and the clauses are depicted that are used to encode the PBC $6*v_1+4*v_2+2*v_3=6$. Sums are written inside the nodes, and \emph{extendable} variables are denoted as $a_i$ for $i\in \{1,2,...,10\}$. Edges with a circle correspond to assigning \false{}. Edges with a line correspond to assigning \true{}.

\begin{figure}
\centering
\includegraphics[width=\textwidth]{Figures/BDDExampleComposition3.png}
\caption{BDD and clauses to encode $6*v_1+4*v_2+2*v_3=6$}
\label{fig:BDDExample}
\end{figure}


\section{Encoding of PBCs using Adder Networks}\label{PBCEncodingAdderNetworks}
Adder Networks are built from Full and Half Adder nodes. A Full Adder (FA) is a node with three inputs and two outputs. A Half Adder (HA) is a node with two inputs and two outputs. The input and output values are binary (\true{}/\false{}). The outputs are name \emph{sum} and \emph{carry}, and their values are computed as follows. If at least two inputs are true, the carry output is true. Otherwise, it is false. If one or three inputs are true, the sum output is true, Otherwise it is false. This behaviour can be encoded into clauses, and as the name implies, it allows Adder Networks to compute sums over binary numbers. Binary numbers are Strings of bits, where each bit is either $1$ (\true) or $0$ (\false). The bit at position $k$ represents a value of $2^k$ and is called $k$-bit (This notation differs from the one used in \cite{Een2006TranslatingPC}). \\

An Adder Network used to sum numbers is explained best layer by layer. When adding binary numbers, the summand $k$-bits get fed into layer $k$. Layer $k$ then also takes the carry values from the previous layer $(k-1)$ as input for its nodes except if $k$ is $0$.  A layer's sum outputs are fed as input to adder nodes in the same layer (potentially requiring additional nodes) until only one input value remains. The remaining value of layer $k$ then corresponds to the value of the $k$-bit of the overall addition result. The carry outputs of a layer are ``carried over" to the next layer, where they become inputs to the adder nodes. The last layer has only one input, and its value gets directly treated as the output of this layer. Figure \ref{fig:AdderNetworkExample} shows an example network that can add three numbers, A, B and C, with values from $0$ to $7$ ($3$ bit long numbers).\\

\begin{figure}[ht!]
\centering
\includegraphics[width=0.72\textwidth]{Figures/AdderNetworkExample2.png}
\caption{Adder Network computing $6 + 4 + 2 = 12$}
\label{fig:AdderNetworkExample}
\end{figure}

As shown in \cite{Een2006TranslatingPC}, Adder Networks can be used to encode PBC. The weights of the PBC are taken as inputs for the network. Every input bit gets associated with the boolean variable that is multiplied by the weight they encode in the PBC. In practice, this is accomplished by feeding the boolean variable into the network directly (in the case of a \true{} bit) or by feeding nothing into the network (in the case of a \false{} bit). So if a boolean variable on the LHS of a PBC is \false{}, all the input bits of the corresponding weights are interpreted as $0$ by the network. The outputs of the adder nodes also get boolean variables assigned, including those outputs corresponding to the bit values of the overall result. These values should be equal to the corresponding bit values of the PBC's RHS as the PBCs we want to encode are equations. To enforce this, the values get compared, and corresponding unit clauses are added to the result set of clauses.\\

The encoding procedure (an adapted version of the algorithm  originally proposed in \cite{Een2006TranslatingPC}) is depicted as pseudocode in Figure \ref{CodeAdderNetwork}. What makes the algorithm presented in \cite{Een2006TranslatingPC} so efficient is that only the needed adder nodes are created, depending on how many \true{} inputs there are in a layer. This dynamic generation keeps the number of adder nodes that actually get transformed into clauses as small as possible. For example, to encode the PBC $6*v_1+4*v_2+2*v_3=6$, only two adder nodes (One Half Adder in Layer-1 and one Full Adder in Layer-2) must be transformed into clauses even though seven adders are needed to add up the numbers $6$, $4$ and $2$ (see Figure \ref{fig:AdderNetworkExample}).

{
\pseudo{}
\renewcommand{\lstlistingname}{Algorithm}
\begin{lstlisting}[frame=single,caption={Pseudocode of PBC encoding using an Adder Network},captionpos=b, label=CodeAdderNetwork, basicstyle=\footnotesize]
AdderNetworkEncoder(PBC):
    clauseSet = empty Set
    Translate PBC weights to binary
    Translate RHS to binary
    bucket = Create empty map, integer --> set
    For every PBC weight:
        For every k-bit of the weight:
            If the bit is true:
                If bucket has no key == k:
                    bucket[k] = new set
                Add variable that is multiplied with current weigth 
                in the PBC to bucket[k]
    networkOut = empty list
    k = 0
    While bucket has keys >= k:
        While bucket has no key == k:
            Append false-variable to networkOut
            k++
        While bucket[k] contains at least 3 variables:
            Create Full Adder with 3 variables of the k-bucket as input
            Add Full Adder clauses to clauseSet
            Create (k+1)-bucket if not present
            Put the Full Adders carry variable in bucket[k+1]
            Put the Full Adders sum variable in bucket[k]
        While bucket[k] contains at least 2 variables:
            Create Half Adder with 2 variables of the k-bucket as input
            Add Half Adder clauses to clauseSet
            Create (k+1)-bucket if not present
            Put the Half Adders carry variable in bucket[k+1]
            Put the Half Adders sum variable in bucket[k]
        Append the one remaining variable of bucket[k] to networkOut
        k++
    While networkOut.length < RHS.length:
        Append false-variable to networkOut
    While networkOut.length > RHS.length:
        Append false-variable to RHS
    For every RHS bit:
        If bit is true:
            Add a clause that contains a positive literal
            of the corresponding networkOut variable to clauseSet
        Else:
            Add a clause that contains a negative literal
            of the corresponding networkOut variable to clauseSet
    Return clauseSet
\end{lstlisting}
}

During algorithm \ref{PBCEncodingAdderNetworks}, Half and Full Adders are transformed into clauses, which can then be added to the result set of clauses that the procedure returns. Assuming the input variables of an adder are $x$, $y$ and $z$, the corresponding output variables are $s$ for the sum and $c$ for the carry. Then the needed implications (as presented in \cite{Een2006TranslatingPC} for Full Adders) to logically describe the adder nodes are as shown in Figure \ref{AdderNetworkClauses}.

\begin{figure}
    \centering
    %\def\arraystretch{1.5}
    \begin{tabular}{l l l}
    %\hline
    %&&\\
    \multicolumn {3}{l}{Full Adder clauses:}\\
     $\{~~x,~~y,~~z,\neg s\}$           & from & $(\neg x   \land \neg y    \land \neg z)   \rightarrow \neg s$ \\
     $\{~~x,\neg y,\neg z,\neg s\}$     & from & $(\neg x   \land ~~y       \land ~~z)      \rightarrow \neg s$ \\
     $\{\neg x,~~y,\neg z,\neg s\}$     & from & $(~~x      \land \neg y    \land ~~z)      \rightarrow \neg s$ \\
     $\{\neg x,\neg y,~~z,\neg s\}$     & from & $(~~x      \land ~~y       \land \neg z)   \rightarrow \neg s$ \\
     $\{\neg x,\neg y,\neg z, ~~s\}$    & from & $(~~x      \land ~~y       \land ~~z)      \rightarrow ~~s$ \\
     $\{\neg x, ~~y, ~~z, ~~s\}$        & from & $(~~x      \land \neg y    \land \neg z)   \rightarrow ~~s$ \\
     $\{~~x, \neg y, ~~z, ~~s\}$        & from & $(\neg x   \land ~~y       \land \neg z)   \rightarrow ~~s$ \\
     $\{~~x, ~~y, \neg z, ~~s\}$        & from & $(\neg x   \land \neg y    \land ~~z)      \rightarrow ~~s$ \\
     $\{\neg x, \neg y, ~~c\}$          & from & $(~~x      \land ~~y)      \rightarrow ~~c$ \\
     $\{\neg x, \neg z, ~~c\}$          & from & $(~~x      \land ~~z)      \rightarrow ~~c$ \\
     $\{\neg y, \neg z, ~~c\}$          & from & $(~~y      \land ~~z)      \rightarrow ~~c$ \\
     $\{~~x, ~~y, \neg c\}$          & from & $(\neg x      \land \neg y)   \rightarrow \neg c$ \\
     $\{~~x, ~~z, \neg c\}$          & from & $(\neg x      \land \neg z)   \rightarrow \neg c$ \\
     $\{~~y, ~~z, \neg c\}$          & from & $(\neg y      \land \neg z)   \rightarrow \neg c$ \\
     &&\\
     \multicolumn {3}{l}{Half Adder clauses:}\\
     $\{~~x, ~~y, \neg s\}$          & from & $(\neg x      \land \neg y)   \rightarrow \neg s$ \\
     $\{~~x, \neg y, ~~s\}$          & from & $(\neg x      \land ~~y)      \rightarrow ~~s$ \\
     $\{\neg x, ~~y, ~~s\}$          & from & $(~~x         \land \neg y)   \rightarrow ~~s$ \\
     $\{\neg x, \neg y, ~~s\}$       & from & $(~~x         \land ~~y)   \rightarrow \neg s$ \\
     $\{~~x, ~~y, \neg c\}$          & from & $(\neg x      \land \neg y)   \rightarrow \neg c$ \\
     $\{~~x, \neg y, \neg c\}$          & from & $(\neg x      \land ~~y)      \rightarrow \neg c$ \\
     $\{\neg x, ~~y, \neg c\}$          & from & $(~~x         \land \neg y)   \rightarrow \neg c$ \\
     $\{\neg x, \neg y, ~~c\}$       & from & $(~~x         \land ~~y)   \rightarrow ~~c$ \\
     %&&\\
     %\hline
    \end{tabular}
    \caption{Clauses for Full and Half Adders.\\
    (Inputs: $x$,$y$,$z$  Sum: $s$  Carry: $c$)}
    \label{AdderNetworkClauses}
\end{figure}

\newpage
\section{Encoding of Sudoku Variants and Constraints}

\subsection{Normal Sudoku}
The normal Sudoku rules as introduced in \ref{NormalSudoku} can be broken down into the following five constraints, which can be encoded into clauses using the variable $s_{x,y,z}$, which is true iff cell $(x,y)$ has value $z$. The following encoding can be seen as a direct encoding using at-least-one and at-most-one clauses and was proposed by \cite{Lynce2006SudokuAsASATProblem} where it is called the minimal encoding:\\


\begin{table}[ht!]
    \centering
    \begin{tabular*}{\textwidth}{l @{\extracolsep{\fill}}  c  c}
        \hline
        \\
        Constraint & Formula & \#Clauses\\
        \\
        \hline
        \\
        At least one number from 1 to 9 appears in each grid cell. & (S-\ref{S-i}) & 81\\
        \\
        Every number appears at most once per row. & (S-\ref{S-ii}) & 2916\\
        \\
        Every number appears at most once per column. & (S-\ref{S-iii}) & 2916\\
        \\
        Every number appears at most once per box. & (S-\ref{S-iv}) and (S-\ref{S-v}) & 2916\\
        \\
        Every cell that contains a hint can only have that value. & (S-\ref{S-vi}) & 1/hint\\
        \\
        \hline
    \end{tabular*}
        \caption{Constraints of Normal Sudoku.}
    \label{constraints:NormalSudoku}
\end{table}
\begin{table}
    \centering
    \begin{tabular*}{\textwidth}{ l @{\extracolsep{\fill}} c}
    \hline
    \\
    $\displaystyle \bigwedge_{x=1}^9 \bigwedge_{y=1}^9 \bigvee_{z=1}^9 s_{x,y,z}$  & \consCount{S} \label{S-\roman{cons}}\\
    \\
    $\displaystyle \bigwedge_{y=1}^9 \bigwedge_{z=1}^9 \bigwedge_{x=1}^9 \bigwedge_{i=x+1}^9 \neg s_{x,y,z} \lor \neg s_{i,y,z}$  & \consCount{S} \label{S-\roman{cons}}\\
    \\
    $\displaystyle \bigwedge_{x=1}^9 \bigwedge_{z=1}^9 \bigwedge_{y=1}^9 \bigwedge_{i=y+1}^9 \neg s_{x,y,z} \lor \neg s_{x,i,z}$  & \consCount{S} \label{S-\roman{cons}}\\
    \\
    $\displaystyle \bigwedge_{z=1}^9 \bigwedge_{i=0}^2 \bigwedge_{j=0}^2 \bigwedge_{x=1}^3 \bigwedge_{y=1}^3 \bigwedge_{k=y+1}^3 \neg s_{(3*i+x),(3*j+y),z} \lor \neg s_{(3*i+x),(3*j+k),z}$  & \consCount{S} \label{S-\roman{cons}}\\
    \\
    $\displaystyle \bigwedge_{z=1}^9 \bigwedge_{i=0}^2 \bigwedge_{j=0}^2 \bigwedge_{x=1}^3 \bigwedge_{y=1}^3 \bigwedge_{k=x+1}^3 \bigwedge_{l=1}^3 \neg s_{(3*i+x),(3*j+y),z} \lor \neg s_{(3*i+k),(3*j+l),z}$  & \consCount{S} \label{S-\roman{cons}}\\
    \\
    $s_{x,y,input_{x,y}}$,  for every $(x,y)$ s.t. input$_{x,y}$ $\neq 0$  & \consCount{S} \label{S-\roman{cons}}\\
    \\
    \hline
    \end{tabular*}
    \caption{Formulae of clauses, Normal Sudoku.}
    \label{formulae:NormalSudoku}
\end{table}


    
\subsection{Anti-Knight}
To encode the Anti-Knight rule, one must ensure that for each grid cell (x,y), it is forbidden to have the same value as its Knight-Neighbours (cells that are a knights distance away). The set of Knight-Neighbours to cell (x,y) can be defined as:
\begin{center}
    $N(x,y) = \{(i,j)~|\textnormal{ cell }(i,j) \textnormal{ one knight distance from cell }(x,y)\}$.
\end{center}

\begin{table}[ht!]
    \centering
    \begin{tabular*}{\textwidth}{l @{\extracolsep{\fill}} c  c}
        \hline
        \\
        Constraint & Formula & \#Clauses\\
        \\
        \hline
        \\
        \makecell[cl]{Cells that are one knight distance apart (neighbours) \\ must have different values.} & (AK-\ref{AK-i}) & 2016\\
        \\
        \hline
    \end{tabular*}
        \caption{Constraints of Anti-Knight rule.}
    \label{constraints:AntiKnight}
\end{table}

\begin{table}
    \centering
    \begin{tabular*}{\textwidth}{ l @{\extracolsep{\fill}} c}
    \hline
    \\
    $\displaystyle \bigwedge_{x=1}^9 \bigwedge_{y=1}^9 \bigwedge_{(i,j)\in N(x,y)} \bigwedge_{z=1}^9 \neg s_{x,y,z} \lor \neg s_{i,j,z}$ &\consCount{AK} \label{AK-\roman{cons}}\\\
    \\
    \hline
    \end{tabular*}
    \caption{Formulae of clause, Anti-Knight rule.}
    \label{formulae:AntiKnight}
\end{table}


\newpage
\subsection{Killer}
\paragraph{Using PBCs:} For every killer cage of the input, we create a list of its cells. From every list, a PBC is created as follows: For every cell (x,y) of a cage, we add $\sum_{i=1}^{9} (s_{x,y,i}*i)$ to the left-hand side of the PBC. The right-hand side of the PBC is set to the target sum that was given as input. The different PBCs (one for every killer cage) can then be encoded into clauses, as explained in \ref{PBCEncodingBDD} and \ref{PBCEncodingAdderNetworks}.

\paragraph{Using PBCs + Combinations:}
The PBC approach can be further optimized, because, given a fixed number of summands, not all values from 1 to 9 can be used to achieve a particular sum. For example, if a cage has a target sum of 8 and consists of three cells, the number of possible value combinations to achieve the target sum is fairly limited. There are only two possible value combinations $1+2+5=8$ and $1+3+4=8$, so the allowed values the cells could take are $1$, $2$, $3$, $4$ and $5$. When constructing the PBC, this knowledge can be used to reduce the number of variable-value products on the left-hand side of the equation. For every cell in a cage, we only add the variable times the corresponding value (to the left-hand side) if the value is an allowed one.

\paragraph{Using Combinations:}
Another possibility is to completely abandon PBCs and exploit that only certain value combinations are possible given a cage with a fixed target sum and fixed number of cells that belong to it. To encode this every combination is given a corresponding variable $s_{v}$, for $v \in V_g  \subset\mathbb{N}$, which is true iff the corresponding combination is used in a certain cage. The set of all cages is annotated as $G$, and the set of all possible combinations to achieve the target sum of a cage $g \in G$ is denoted as $C_g$.

\begin{table}[ht!]
    \centering
    \begin{tabular*}{\textwidth}{l @{\extracolsep{\fill}} c}
        \hline
        \\
        Constraint & Formula\\
        \\
        \hline
        \\
        \makecell[cl]{For every Cage $g \in G$ and possible combination $c \in C_g$ it\\
        holds that, either the cage's target sum is not achieved using\\
        combination $c$ or every cage cell contains at least one value\\
        of the combination $c$.} & (K-\ref{K-i})\\
        \\
        In every Cage $g \in G$ at least one combination $c \in C_g$ is used. & (K-\ref{K-ii})\\
        \\
        In every Cage $g \in G$ at most one combination $c \in C_g$ is used. & (K-\ref{K-iii})\\
        \\
        \makecell[cl]{Every value from 1 to 9 appears at most once within the\\
        cells of a cage $g \in G$.} & (K-\ref{K-iv})\\
        \\
        \hline
    \end{tabular*}
        \caption{Constraints of Killer Sudoku rules.}
    \label{tab:Killer}
\end{table}

\begin{table}
    \centering
    \begin{tabular*}{\textwidth}{ m{70mm} l @{\extracolsep{\fill}} c}
    \hline
    \\
    $\displaystyle \bigwedge_{g:G} \bigwedge_{c:C_g} \bigwedge_{(x,y):g} -s_v \lor \bigvee_{z:c}  s_{x,y,z}$ & & \consCount{K} \label{K-\roman{cons}}\\
    \\
    $\displaystyle \bigwedge_{g:G} \bigvee_{v:V_g} s_{v}$ & & \consCount{K} \label{K-\roman{cons}}\\
    \\
    $\displaystyle \bigwedge_{g:G} \bigwedge_{v':V_g} \bigwedge_{v'':V_g} \neg s_{v'} \lor \neg s_{v''}$  & with $v' < v''$ & \consCount{K} \label{K-\roman{cons}}\\
    \\
    $\displaystyle \bigwedge_{g:G} \bigwedge_{(x_i,y_i):g} \bigwedge_{(x_j,y_j):g} \bigwedge_{z=1}^{9} \neg x_i y_i z \lor \neg x_j y_j z$ & with $(x_i,y_i) \neq (x_j,y_j)$ &\consCount{K} \label{K-\roman{cons}}\\
        \\
    \hline
    \end{tabular*}
    \caption{Formulae of clauses, Killer Sudoku rules.}
    \label{formulae:KillerSudoku}
\end{table}

\clearpage

\subsection{Arrowheads}\label{Encoding:Arrowheads}
Arrowheads demand a total or partial order between two cells. The sets of all tuples of two cells under total or partial order are respectively noted as $TO$ and $PO$. The first cell of such a tuple  $((x_1,y_1), (x_2,y_2)) \in TO \cup PO$ shall have a smaller (or equal) value than the second one $(x_2,y_2)$. To enforce this, the constraints shown in \ref{tab:ArrowHeads} are encoded into clauses using the support encoding. In formula AH-\ref{AH-i}, the values of $z_1$ go from $1$ to $9$, which ensures that the first cell of a tuple can not have a value of 9. In formula AH-\ref{AH-ii}, on the other hand, $z_1$ iterates from $2$ to $9$ because if the first cell has a value of $1$, all values from $1$ to $9$ are allowed for the second cell.\\

\begin{table}[ht!]
    \centering
    \begin{tabular*}{\textwidth}{l @{\extracolsep{\fill}} c}
        \hline
        \\
        Constraint & Formula\\
        \\
        \hline
        \\
        \makecell[cl]{For every $((x_1,y_1), (x_2,y_2)) \in TO$, either cell $(x_1,y_1)$ has not\\
        value $z_1\in \{1,..,9\}$ or cell $(x_2,y_2)$ has value $z_2 > z_1, z_2 \in \{2,..,9\}$.} & (AH-\ref{AH-i})\\
        \\
        \makecell[cl]{For every $((x_1,y_1), (x_2,y_2)) \in PO$, either cell $(x_1,y_1)$ has not\\
        value $z_1\in \{2,..,9\}$ or cell $(x_2,y_2)$ has value $z_2 \geq z_1, z_2 \in \{1,..,9\}$.} & (AH-\ref{AH-ii})\\
        \\
        \hline
    \end{tabular*}
        \caption{Constraints of Arrowhead rules.}
    \label{tab:ArrowHeads}
\end{table}
\begin{table}[h!]
    \centering
    \begin{tabular*}{\textwidth}{ l l @{\extracolsep{\fill}} c}
    \hline
    \\
    $\displaystyle \bigwedge_{((x_1,y_1),(x_2,y_2)):TO} \bigwedge_{z_1=1}^{9} \neg s_{x_1,y_1,z_1} \lor \bigvee_{z_2=z_1+1}^{9} s_{x_2,y_2,z_2}$ & & \consCount{AH} \label{AH-\roman{cons}}\\
    \\
    $\displaystyle \bigwedge_{((x_1,y_1),(x_2,y_2)):PO} \bigwedge_{z_1=2}^{9} \neg s_{x_1,y_1,z_1} \lor \bigvee_{z_2=z_1}^{9} s_{x_2,y_2,z_2}$ & & \consCount{AH} \label{AH-\roman{cons}}\\
        \\
    \hline
    \end{tabular*}
    \caption{Formulae of clauses, Arrowhead rules.}
    \label{formulae:Arrowhead}
\end{table}

\clearpage

\subsection{Thermometers (Hidden)}
As the Thermometer rule also demands a total (or partial) order between the cells of the thermometers, one can reuse the rules and formulas shown for Arrowheads in \ref{Encoding:Arrowheads}. Starting from a Thermometers bulb, one can conceptually place Arrowheads between every cell pair along the Thermometer, enforcing a total (or partial) order between all cells.\todoMissing{Figure of Arrowheads on Thermometer.}\\

Further constraints and variables are needed for the more complicated puzzle, where the solver must also deduce the thermometer positions.  
When encoding this puzzle, we distinguish two main cases: if the entire thermometer number $t$ fits inside the $9\times9$ grid when its bulb is placed at $(x,y)$, the clauses of TH-\ref{TH-iv}, TH-\ref{TH-v} and TH-\ref{TH-vi} are added. Otherwise the clauses of TH-\ref{TH-vii} are added.
In the formulae the following annotations are used:

\begin{table}[!ht]
    \centering
    \begin{tabular}{l c l}
    \hline
    \\
    Notation: &&\\
    \\
    \hline
    \\
    $T$                     &= &list of given thermometers\\
    $T.size$                &= &number of given thermometers\\
    $T[t]$                  &= &thermometer number t in list of all thermometers\\
    $T[t].size$             &= &number of cells in thermometer number $t$\\
    $\rchi(t,x,y,d)$           &= &$x$-coordinate of the cell that is at depth $d$ of the\\
                            &  &thermometer number $t$ if its bulb is placed in cell\\
                            &  &$(x,y)$\\
    $\Upsilon(t,x,y,d)$           &= &$y$-coordinate of the cell that is at depth $d$ of the\\
                            &  &thermometer number $t$ if its bulb is placed in cell\\
                            &  &$(x,y)$\\
    $Arrowhead(t,x,y,d)$    &= &set of clauses needed to encode that the cell at\\
                            &  &depth $d$ of thermometer number $t$ is smaller than\\
                            &  &the cell at depth $d+1$, given the thermometer\\
                            &  &has its bulb in cell $(x,y)$.\\
    \\
    \hline
    \\
    Variables:&&\\
    $s_{t,d,x,y}$           &: &is true if and only if the cell at depth $d$ of\\
                            &  &thermometer number $t$ is located in grid cell\\
                            &  &$(x,y)$\\
    \\
    \hline
\end{tabular}
    \caption{Notations and variables, Thermometers Hidden rules.}
    \label{notation:ThermometersHidden}
\end{table}





\begin{table}[ht!]
    \centering
    \begin{tabular*}{\textwidth}{l @{\extracolsep{\fill}} c}
        \hline
        \\
        Constraint & Formula\\
        \\
        \hline
        \\
        \makecell[cl]{Every cell of every thermometer is located in at least one\\
        grid cell.} & (TH-\ref{TH-i})\\
        \\
        \makecell[cl]{At most one thermometer cell is located in one grid cell.} & (TH-\ref{TH-ii}) and (TH-\ref{TH-iii})\\
        \\
        \makecell[cl]{If the cell at depth $d$ of thermometer number $t$ is located\\
        in grid cell $(x, y)$, then the thermometer cell at depth\\
        $d+1$ must be located next to it (corresponding to the\\
        thermometer's shape), except if $d$ = T[t].size.} & (TH-\ref{TH-iv})\\
        \\
        \makecell[cl]{If the cell at depth $d$ of thermometer number $t$ is located\\
        in grid cell $(x, y)$, then the thermometer cell at depth\\
        $d-1$ must be located next to it (corresponding to the\\
        thermometer's shape), except if $d$ = 0.} & (TH-\ref{TH-v})\\
        \\
        \makecell[cl]{If the cell at depth $d$ of thermometer number $t$ is located\\
        in grid cell $(x, y)$, then the cell value of $(x,y)$ must be\\
        lower than that of the thermometer cell at depth $d+1$.} & (TH-\ref{TH-vi})\\
        \\
        \makecell[cl]{If not the entire thermometer with number t is inside\\
        the $9 \times 9$ when its bulb is placed at cell $(x,y)$, then the\\
        cell at depth $d$ of thermometer number $t$ can not be\\
        located at the cell it covers in this situation.} & (TH-\ref{TH-vii})\\
        \hline
    \end{tabular*}
        \caption{Constraints of the Thermometers-Hidden rules.}
    \label{Constraints:Thermometers}
\end{table}
\begin{table}[ht!]
    \begin{tabular*}{\textwidth}{ l l @{\extracolsep{\fill}} c}
    \hline
    \\
    $\displaystyle \bigwedge_{t=1}^{T.size} \bigwedge_{d=1}^{T[t].size} \bigvee_{x=1}^{9} \bigvee_{y=1}^{9} s_{t,d,x,y}$ & & \consCount{TH} \label{TH-\roman{cons}}
    \\
    \\
    $\displaystyle \bigwedge_{t=1}^{T.size} \bigwedge_{d=1}^{T[t].size} \bigwedge_{k=d+1}^{T[t].size} \bigwedge_{x=1}^{9} \bigwedge_{y=1}^{9}  \neg s_{t,d,x,y} \lor \neg s_{t,k,x,y}$ & & \consCount{TH} \label{TH-\roman{cons}}
    \\
    \\
    $\displaystyle \bigwedge_{t=1}^{T.size} \bigwedge_{k=t+1}^{T.size} \bigwedge_{d=1}^{T[t].size}  \bigwedge_{l=1}^{T[t].size} \bigwedge_{x=1}^{9} \bigwedge_{y=1}^{9} \neg s_{t,d,x,y} \lor \neg s_{k,l,x,y}$ & & \consCount{TH} \label{TH-\roman{cons}}
    \\
    \\
    $\displaystyle \bigwedge_{t=1}^{T.size} \bigwedge_{x=1}^{9} \bigwedge_{y=1}^{9} \bigwedge_{d=1}^{T[t].size-1} \neg s_{t,d,\rchi(t,x,y,d), \Upsilon(t,x,y,d)} \lor s_{t,d+1,\rchi(t,x,y,d+1), \Upsilon(t,x,y,d+1)}$ & & \consCount{TH} \label{TH-\roman{cons}}
    \\
    \\
    $\displaystyle \bigwedge_{t=1}^{T.size} \bigwedge_{x=1}^{9} \bigwedge_{y=1}^{9} \bigwedge_{d=1}^{T[t].size-1} s_{t,d,\rchi(t,x,y,d), \Upsilon(t,x,y,d)} \lor \neg s_{t,d+1,\rchi(t,x,y,d+1), \Upsilon(t,x,y,d+1)}$ & & \consCount{TH} \label{TH-\roman{cons}}
    \\
    \\
    $\displaystyle \bigwedge_{t=1}^{T.size} \bigwedge_{x=1}^{9} \bigwedge_{y=1}^{9} \bigwedge_{d=1}^{T[t].size-1} \neg s_{t,d,\rchi(t,x,y,d), \Upsilon(t,x,y,d)} \lor Arrowhead(t,x,y,d)$ & & \consCount{TH} \label{TH-\roman{cons}}
    \\
    \\
    $\displaystyle \bigwedge_{t=1}^{T.size} \bigwedge_{x=1}^{9} \bigwedge_{y=1}^{9} \bigwedge_{d=1}^{T[t].size} \neg s_{t,d,\rchi(t,x,y,d), \Upsilon(t,x,y,d)}$ ~~~~~~\textnormal{with} $\rchi(), \Upsilon() \in [1,9]$ & & \consCount{TH} \label{TH-\roman{cons}}\\
        \\
    \hline
\end{tabular*}
    \caption{Formulae of clauses, Thermometers-Hidden rules.}
    \label{formulae:thermometers}
\end{table}

\FloatBarrier
\newpage
\subsection{Sandwich Sum}
Sandwich Sums can be given for rows and columns. We will elaborate on the row constraints from which the formulas for the column constraints can be derived by swapping x and y, respectively. Similar to the Killer Sudoku rules, multiple optimizations can be (incrementally) made when encoding the Sandwich Sum rules.

\paragraph{Using PBCs:} Assuming that all lengths of sandwiches are possible to achieve a certain sum. We must encode all possible sandwich lengths ($0$ to $7$) and all corresponding positions for the cells with values $1$ and $9$. Also, we encode a PBC for every possible sandwich length and position. The LHS of said PBC is $\sum_{i=1}^{9} (s_{x,y,i}*i)$, and the RHS is set to the current rows sandwich sum. Since the Sandwich can only be at one position in a row at once, only one of all the PBC will be true. To ensure that a PBC's clauses are satisfied if the sandwich is not in its corresponding position, we add the negative literal of the corresponding $s_v$ to each clause (stylized in formula SW-\ref{SW-vii}). To encode the Sandwich Sum rules for rows without the incorporation of further knowledge, the following constraints/formulas are used: SW-\ref{SW-i}, SW-\ref{SW-ii}, SW-\ref{SW-vii}, SW-\ref{SW-x} and SW-\ref{SW-xi}.

\paragraph{Using PBCs + Combinations for Lengths:} Given the Sandwich Sum of a row, one can already make statements about the number of cells involved in the sum. For example, given a sandwich sum of $8$, one can conclude that the sandwich has either length $3$, $2$, or $1$ (contains $3$, $2$ or $1$ cells, not counting the border cells with values $1$ and $9$). This already reduces the number of combinations of sandwich lengths and positions and can be utilized to encode the rules using the constraints/formulas: SW-\ref{SW-iii}, SW-\ref{SW-iv}, SW-\ref{SW-v}, SW-\ref{SW-vi}, SW-\ref{SW-viii}, SW-\ref{SW-x} and SW-\ref{SW-xi}.

\paragraph{Using Combinations for PBCs + Combinations for Lengths:} Given the Sandwich Sum of a row, one can not only make statements about the number of cells involved in the sum but also about the possible values of these cells given a certain sandwich length (Similar to the possible cell values in a killer cage given the cages sum and size). This reduces the number of summands of the LHS of PBCs and can be combined with the knowledge about possible sandwich lengths. To encode the sandwich sum rules using this additional knowledge, the following constraints/formulas: are used: SW-\ref{SW-iii}, SW-\ref{SW-iv}, SW-\ref{SW-v}, SW-\ref{SW-vi}, SW-\ref{SW-ix}, SW-\ref{SW-x}, SW-\ref{SW-xi} and SW-\ref{SW-xii}.

\begin{table}
    \centering
    \begin{tabular*}{\textwidth}{l @{\extracolsep{\fill}} c}
        \hline
        \\
        Constraint on rows & Formula\\
        \\
        \hline
        \\
        \makecell[cl]{If the sandwich of a row is at a certain position, the cells\\
        with values 1 and 9 must be positioned accordingly.} & (SW-\ref{SW-i}) and (SW-\ref{SW-ii})\\
        \\
        \makecell[cl]{If the sandwich of a row (with compatible length regarding\\
        the sandwich sum) is at a certain position, the cells with\\
        values 1 and 9 must be positioned accordingly.} & (SW-\ref{SW-iii}) and (SW-\ref{SW-iv})\\
        \\
        \makecell[cl]{The cells with values 1 and 9 can not have a certain\\
        number of cells between them, if the values from these\\
        number of cells can not add up to the rows sandwich sum.} & (SW-\ref{SW-v}) and SW-\ref{SW-vi})\\
        \\
        \makecell[cl]{If the sandwich of a row is at a certain position, then the\\
        corresponding cells must add up to the row's sandwich\\
        sum.} & (SW-\ref{SW-vii})\\
        \\
        \makecell[cl]{If the sandwich of a row (with compatible length regarding\\
        the sandwich sum) is at a certain position, then the\\
        corresponding cells must add up to the row's sandwich\\
        sum.} & (SW-\ref{SW-viii})\\
        \\
        \makecell[cl]{If the sandwich of a row (with compatible length regarding\\
        the sandwich sum) is at a certain position, then the\\
        corresponding cells must add up to the row's sandwich sum\\
        using only compatible values regarding the sandwich sum.} & (SW-\ref{SW-ix})\\
        \\
        \makecell[cl]{The sandwich must be in at least one position.} & (SW-\ref{SW-x})\\
        \\
        \makecell[cl]{The sandwich must be in at most one position.} & (SW-\ref{SW-xi})\\
        \\
        \makecell[cl]{If the sandwich of a row is at a certain position, the cell\\
        values of the cells that are added up to the sandwich sum\\
        can not be incompatible regarding the sandwich sum.} & (SW-\ref{SW-xii})\\
        \\
        \hline
    \end{tabular*}
        \caption{Constraints of Sandwich Sum rules.}
    \label{Constraints:SandwichSum}
\end{table}


\begin{table}
    \centering
    \begin{tabular}{l c l}
    \hline
    \\
    Notation: &&\\
    \\
    \hline
    \\
    $PBC(x_1,x_2,y,sum)$    &= &set of clauses needed to encode that the cells\\
                            &  &between $(x_1,y)$ and $(x_2,y)$ have values $\in \{1,...,9\}$\\
                            &  & that add up to $sum$.\\
    $PBC(x_1,x_2,y,sum,Val)$&= &set of clauses needed to encode that the cells\\
                            &  &between $(x_1,y)$ and $(x_2,y)$ have values $\in \textit{Val}$\\
                            &  &that add up to $sum$.\\
    $L(y)$                  &= &set of numbers corresponding to the sizes of cell\\
                            &  &sets for which it is possible to achieve the\\
                            &  &sandwich sum of row $y$, given that each cell can\\
                            &  &only have values from 1 to 9. $L(y)\subseteq \{0,1,...,7\}$\\
    $\bar{L}(y)$            &= &$\{0,...,7\}\setminus L(y)$\\
    $Z(y,l)$                &= &set of integer values for which it is possible to\\
                            &  &achieve the sandwich sum of row $y$ if $l$ different\\
                            &  &one of them are added.  $Z(y,l)\subseteq \{1,...,9\}$\\
    $\bar{Z}(y,l)$          &= &$\{1,...,9\}\setminus Z(y,l)$\\
    $\mathcal{S}(x,y,l)$    &= &corresponding $s_v$\\
    \\
    \hline
    \\
    Variables:&&\\
    \\
    \hline
    \\
    $s_v$                   &= &dynamically assigned variable for each\\
                            &  &combination of $x$, $y$ and $l$ that is true if the\\
                            &  &sandwich of row $y$ has its borders in cell (x,y)\\
                            &  &and cell (x+1+l,y). $v \in V_y \subset \mathbb{N}$\\
    \\    
    \hline
    \end{tabular}
    \caption{Notations and variables, Sandwich Sum rules.}
    \label{notation:SandwichSum}
\end{table}




\begin{table}[ht!]
    \begin{tabular*}{\textwidth}{ l l @{\extracolsep{\fill}} c}
    \hline
    \\
    $\displaystyle \bigwedge_{y=1}^{9} \bigwedge_{l=0}^{7} \bigwedge_{x=1}^{9}  \neg s_{x,y,1} \lor \neg s_{(x+l+1),y,9} \lor \mathcal{S}(x,y,l)$ & & \consCount{SW} \label{SW-\roman{cons}}\\
    \\
    $\displaystyle \bigwedge_{y=1}^{9} \bigwedge_{l=0}^{7} \bigwedge_{x=1}^{9}  \neg s_{x,y,9} \lor \neg s_{(x+l+1),y,1} \lor \mathcal{S}(x,y,l)$ & & \consCount{SW} \label{SW-\roman{cons}}\\
    \\
    $\displaystyle \bigwedge_{y=1}^{9} \bigwedge_{l:L(y)} \bigwedge_{x=1}^{9}  \neg s_{x,y,1} \lor \neg s_{(x+l+1),y,9} \lor \mathcal{S}(x,y,l)$ & & \consCount{SW} \label{SW-\roman{cons}}\\
    \\
    $\displaystyle \bigwedge_{y=1}^{9} \bigwedge_{l:L(y)} \bigwedge_{x=1}^{9}  \neg s_{x,y,9} \lor \neg s_{(x+l+1),y,1} \lor \mathcal{S}(x,y,l)$ & & \consCount{SW} \label{SW-\roman{cons}}\\
    \\
    $\displaystyle \bigwedge_{y=1}^{9} \bigwedge_{l:\bar{L}(y)} \bigwedge_{x=1}^{9-l-1}  \neg s_{x,y,1} \lor \neg s_{(x+l+1),y,9}$ & & \consCount{SW} \label{SW-\roman{cons}}\\
    \\
    $\displaystyle \bigwedge_{y=1}^{9} \bigwedge_{l:\bar{L}(y)} \bigwedge_{x=1}^{9-l-1}  \neg s_{x,y,9} \lor \neg s_{(x+l+1),y,1}$ & & \consCount{SW} \label{SW-\roman{cons}}\\
    \\
    $\displaystyle \bigwedge_{y=1}^{9} \bigwedge_{l=0}^{7} \bigwedge_{x=1}^{9}  PBC(x,x+l+1,y,sum) \lor \neg \mathcal{S}(x,y,l)$ & & \consCount{SW} \label{SW-\roman{cons}}\\
    \\
    $\displaystyle \bigwedge_{y=1}^{9} \bigwedge_{l:L(y)} \bigwedge_{x=1}^{9}  PBC(x,x+l+1,y,sum) \lor \neg \mathcal{S}(x,y,l)$ & & \consCount{SW} \label{SW-\roman{cons}}\\
    \\
    $\displaystyle \bigwedge_{y=1}^{9} \bigwedge_{l:L(y)} \bigwedge_{x=1}^{9}  PBC(x,x+l+1,y,sum, Z(y,l)) \lor \neg \mathcal{S}(x,y,l)$ & & \consCount{SW} \label{SW-\roman{cons}}\\
    \\
    $\displaystyle \bigwedge_{y=1}^{9} \bigvee_{v:V_y} s_v$ & & \consCount{SW} \label{SW-\roman{cons}}\\
    \\
    $\displaystyle \bigwedge_{y=1}^{9} \bigwedge_{v':V_y} \bigwedge_{v'':V_y} \neg s_{v'} \lor \neg s_{v''}$ & with $v' < v''$ & \consCount{SW} \label{SW-\roman{cons}}\\
    \\
    $\displaystyle \bigwedge_{y=1}^{9} \bigwedge_{l:L(y)} \bigwedge_{x=1}^{9}  \bigwedge_{z:\bar{Z}(y,l)} \bigwedge_{x'=x+1}^{x+l} \neg s_{x',y,z} \lor \neg \mathcal{S}(x,y,l)$ & & \consCount{SW} \label{SW-\roman{cons}}\\
    \\
    \hline
\end{tabular*}
    \caption{Formulae of clauses, Sandwich Sum rules.}
    \label{formulae:SandwichSum}
\end{table}