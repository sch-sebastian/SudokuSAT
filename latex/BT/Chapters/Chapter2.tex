% !TEX root = ../Thesis.tex
%\theoremstyle{definition}
\newtheorem{definition}{Definition}

%\renewcommand{\thethm}{\arabic{thm}}% Remove subsection from theorem counter representation


\chapter{Background}
Before we explain different Sudoku Variants and how they can be encoded so that a computer can solve them, we first want to give some background knowledge and definitions needed to understand the tools and formalisms that are used to achieve such an encoding.

\section{Propositional Logic}

\paragraph{Atoms}
(also called \emph{Atomic Propositions} or \emph{Literals}) are the smallest units used in Propositional Logic, and must have a truth value of true or false.


\paragraph{Formulas} are compositions of one or multiple \emph{Atoms} and can be defined recursively:\\
Every \emph{Atom} is also a \emph{Formula}.
If $\varphi$ is a \emph{Formula}, then so is its negation $\neg\varphi$.
If $\varphi$ and $\psi$ are \emph{Formulas}, then so is the conjunction $\varphi \land \psi$.
If $\varphi$ and $\psi$ are \emph{Formulas}, then so is the disjunction $\varphi \lor \psi$.


\paragraph{Interpretations} or truth assignments, are functions that assigns truth values to \emph{Atoms} $\mathcal{I}: A \rightarrow \{0,1\}$. A \emph{Formula} $\varphi$ holds (is true) under an \emph{Interpretation} $\mathcal{I}$ (written $\mathcal{I} \models \varphi$) following the semantical rules:
\begin{center}
    \begin{tabular}{ l l l }
    $\mathcal{I} \models a$ & iff & $\mathcal{I}(a) = 1$\\
    $\mathcal{I} \models \neg a$ & iff & not $\mathcal{I} \models a$\\
    $\mathcal{I} \models (\varphi \land \psi)$ & iff & $\mathcal{I} \models \varphi$ and $\mathcal{I} \models \psi$\\
    $\mathcal{I} \models (\varphi \lor \psi)$ & iff & $\mathcal{I} \models \varphi$ or $\mathcal{I} \models \psi$\\
\end{tabular}
\end{center}

An \emph{Interpretation} for which a formula $\varphi$ hold is called a \textbf{Model} of $\varphi$.

\paragraph{Equivalence}
Two \emph{Formulas} $\varphi$ and $\psi$ are \emph {logically equivalent} ($\varphi \equiv \psi$) if it holds for all \emph{Interpretations} $\mathcal{I}$ that, $\varphi$ holds under $\mathcal{I}$ if and only if $\psi$ holds under $\mathcal{I}$.

\paragraph{Implication / Biconditional}
As one might have noticed, \emph{Implication} ($\rightarrow$) and \emph{Biconditional} ($\leftrightarrow$) have not been mentioned in the definition of \emph{Formulas} as they are abbreviations for more extended \emph{Formulas} that use $\lor$, $\land$ and $\neg$.
\begin{center}
\begin{tabular}{ l l l }
    $(\varphi \rightarrow \psi)$ & $\equiv$ & $(\neg\varphi \lor \psi)$\\
    $(\varphi \leftrightarrow \psi)$ & $\equiv$ & $(\neg\varphi \lor \psi) \land (\neg\psi \lor \varphi)$\\
\end{tabular}
\end{center}

\paragraph{Clause}
A \emph{Clause} is a disjunction of \emph{Atoms}. A \emph{Formula} that is a \emph{Clause} becomes true for \emph{Interpretation} $\mathcal{I}$ if one of its \emph{Atoms} gets assigned to true.

\paragraph{Conjunctive Normal Form}
A \emph{Formula} is said to be in \emph{Conjunctive Normal Form} (CNF) if it is a conjunction of disjunctions of \emph{Atoms}. 
By example given the \emph{Atoms} \emph{a}, \emph{b}, \emph{c} and \emph{d} the \emph{Formulas} $\varphi$, $\psi$ and $\varrho$ are in CNF:
\begin{center}
    \begin{tabular}{ l l l }
    $\varphi$ & $\equiv$ & $(a)$\\
    $\psi$ & $\equiv$ & $(a \land b)$\\
    $\varrho$ & $\equiv$ & $((a \lor b) \land (c \lor d))$\\
\end{tabular}
\end{center}
Also, it holds that every formula can be brought into CNF \cite{ArtificialAModernApproach}.\todo{other reference?}